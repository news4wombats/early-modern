\documentclass[]{article}
\usepackage{hyperref,setspace,graphicx,fancyhdr,multicol,needspace}
\usepackage[width=6in,height=9.5in,top=.5in,centering]{geometry}
% changes font to TeX Gyre Schola (Century Schoolbook)
\usepackage{tgschola}\usepackage[T1]{fontenc}

\setlength{\columnsep}{.3 in}


\hypersetup{pdfinfo={Title={Treatise Concerning the Principles of Human Knowledge}, Author={George Berkeley}}, linkbordercolor = 1 0 0, pdfborder = 0 0 0, colorlinks = false, breaklinks = true, linkcolor = black}  

\newcounter{authornote}[page]
\newcommand*{\authornote}[1]{\renewcommand{\thefootnote}{\fnsymbol{footnote}}\stepcounter{authornote}\footnote[\value{authornote}]{#1}\renewcommand{\thefootnote}{\arabic{footnote}}}

\newcommand*{\hbreak}{\par\noindent\begin{tabular*}{\linewidth}{c}\hline\hline\end{tabular*}\par}

\newcommand*{\authortitle}[1]{\medskip\centerline{\Huge\sc #1}\bigskip}
\newcommand*{\itemtitle}[1]{\setstretch{1.8}\pagebreak[2]\begin{center}{\LARGE\sc #1}\end{center}\setstretch{1.2}}
\newcommand*{\itemsubtitle}[1]{\begin{center}\emph{#1}\end{center}}
\newcommand*{\itemcomment}[1]{\noindent\emph{#1}}
\newcommand*{\itemsection}[1]{\needspace{36pt}\begin{center}\addcontentsline{toc}{section}{#1}\textbf{\Large #1}\nopagebreak\stepcounter{section}\end{center}}
\newcommand*{\itemsubsection}[1]{\begin{center}\addcontentsline{toc}{subsection}{#1}\textbf{#1}\end{center}}

\renewcommand{\thesection}{}

\renewcommand\headrulewidth{0pt}

%\newenvironment{sectionbody}{\begin{multicols}{2}}{\end{multicols}}
\newenvironment{sectionbody}{}{}



\begin{document}

% David R. Wilkins
% School of Mathematics, Trinity College, Dublin 2, Ireland
% (dwilkins@maths.tcd.ie)

\authortitle{George Berkeley}
%\addcontentsline{toc}{section}{Berkeley's \emph{Principles}}

%The biography is taken from Lisa Downing's entry on Berkeley in the \emph{Stanford Encyclopedia of Philosophy}:

%\url{http://plato.stanford.edu/entries/berkeley/}

\noindent
This largely text is based on the Plain \TeX\ edition by David R. Wilkins (Trinity College, Dublin; 2002) which follows the 1734 edition published by Jacob Tonson. The Preface included here was omitted from that edition. This document was prepared by \href{https://www.fecundity.com}{P.D. Magnus}.
\itemtitle{Treatise Concerning the Principles of Human Knowledge}
\itemsubtitle{Wherein the Chief Causes of Error and Difficulty in the Sciences, with the grounds of Scepticism, Atheism, and Irreligion, are inquired into.}

\setstretch{1}


\setcounter{tocdepth}{1}
\tableofcontents

\pagestyle{fancy}
\lfoot{\thepage}
\cfoot{}
\rfoot{\sc Berkeley's {Principles}}

\setstretch{1.2}

\bigskip



\itemsection{The Preface}
\begin{sectionbody}
%\emph
{What I here make public has, after a long and scrupulous inquiry, seem'd to me evidently true, and not unuseful to be known, particularly to those who are tainted with scepticism, or want a demonstration of the existence and immateriality of {\sc God}, or the natural immortality of the soul. Since I do not think my self any farther concerned for the success of what I have written, than as it is agreeable to \emph{truth}. But to the end \emph{this} may not suffer, I make it my request that the reader suspend his judgement, till he has once, \emph{at least}, read the whole through with that degree of attention and thought which the subject matter shall seem to deserve. For as there are some passages that, taken by themselves, are very liable (nor could it be remedied) to gross misinterpretation, and to be charged with the most absurd consequences, which, nevertheless, upon an entire perusal will appear not to follow from them: so likewise, though the whole should be read over, yet, if this be done transiently, 'tis very probable my sense may be mistaken; but to a thinking reader, I flatter my self, it will be throughout clear and obvious. As for the characters of novelty and singularity, which some of the following notions seem to bear, 'tis, I hope, needless to make an apology on that account. He must surely be either very weak, or very little acquainted with the sciences, who shall reject a truth, that is capable of demonstration, for no other reason but because it's newly known and contrary to the prejudices of mankind. Thus much I thought fit to premise, in order to prevent, if possible, the hasty censures of a sort of men, who are too apt to condemn an opinion before they rightly comprehend it.}

\end{sectionbody}

\newpage

\itemsection{Introduction}


\begin{sectionbody}

\paragraph{1.} PHILOSOPHY being nothing else but the study of Wisdom and Truth,
it may with reason be expected, that those who have spent most
Time and Pains in it should enjoy a greater calm and serenity of
Mind, a greater clearness and evidence of Knowledge, and be less
disturbed with Doubts and Difficulties than other Men.  Yet so it
is we see the Illiterate Bulk of Mankind that walk the High-road
of plain, common Sense, and are governed by the Dictates of
Nature, for the most part easy and undisturbed.  To them nothing
that's familiar appears unaccountable or difficult to comprehend.
They complain not of any want of Evidence in their Senses, and
are out of all danger of becoming \emph{Sceptics}.  But no
sooner do we depart from Sense and Instinct to follow the Light
of a Superior Principle, to reason, meditate, and reflect on the
Nature of Things, but a thousand Scruples spring up in our Minds,
concerning those Things which before we seemed fully to
comprehend.  Prejudices and Errors of Sense do from all Parts
discover themselves to our view; and endeavouring to correct
these by Reason we are insensibly drawn into uncouth Paradoxes,
Difficulties, and Inconsistencies, which multiply and grow upon
us as we advance in Speculation; till at length, having wander'd
through many intricate Mazes, we find our selves just where we
were, or, which is worse, sit down in a forlorn Scepticism.



\paragraph{2.} The cause of this is thought to be the Obscurity of things, or
the natural Weakness and Imperfection of our Understandings.  It
is said the Faculties we have are few, and those designed by
Nature for the Support and Comfort of Life, and not to penetrate
into the inward Essence and Constitution of Things.  Besides, the
Mind of Man being Finite, when it treats of Things which partake
of Infinity, it is not to be wondered at, if it run into
Absurdities and Contradictions; out of which it is impossible it
should ever extricate it self, it being of the nature of Infinite
not to be comprehended by that which is Finite.



\paragraph{3.} But perhaps we may be too partial to our selves in placing the
Fault originally in our Faculties, and not rather in the wrong
use we make of them.  It is a hard thing to suppose, that right
Deductions from true Principles should ever end in Consequences
which cannot be maintained or made consistent.  We should believe
that God has dealt more bountifully with the Sons of Men, than to
give them a strong desire for that Knowledge, which he had placed
quite out of their reach.  This were not agreeable to the wonted,
indulgent Methods of Providence, which, whatever Appetites it may
have implanted in the Creatures, doth usually furnish them with
such means as, if rightly made use of, will not fail to satisfy
them.  Upon the whole, I am inclined to think that the far
greater Part, if not all, of those Difficulties which have
hitherto amus'd Philosophers, and block'd up the way to
Knowledge, are intirely owing to our selves.  That we have first
rais'd a Dust, and then complain, we cannot see.



\paragraph{4.} My Purpose therefore is, to try if I can discover what those
Principles are, which have introduced all that Doubtfulness and
Uncertainty, those Absurdities and Contradictions into the
several Sects of Philosophy; insomuch that the Wisest Men have
thought our Ignorance incurable, conceiving it to arise from the
natural dulness and limitation of our Faculties.  And surely it
is a Work well deserving our Pains, to make a strict inquiry
concerning the first Principles of \emph{Humane Knowledge}, to
sift and examine them on all sides: especially since there may be
some Grounds to suspect that those Lets and Difficulties, which
stay and embarass the Mind in its search after Truth, do not
spring from any Darkness and Intricacy in the Objects, or natural
Defect in the Understanding, so much as from false Principles
which have been insisted on, and might have been avoided.



\paragraph{5.} How difficult and discouraging soever this Attempt may seem, when
I consider how many great and extraordinary Men have gone before
me in the same Designs: Yet I am not without some Hopes, upon the
Consideration that the largest Views are not always the Clearest,
and that he who is Short-sighted will be obliged to draw the
Object nearer, and may, perhaps, by a close and narrow Survey
discern that which had escaped far better Eyes.



\paragraph{6.} In order to prepare the Mind of the Reader for the easier
conceiving what follows, it is proper to premise somewhat, by way
of Introduction, concerning the Nature and Abuse of Language.
But the unraveling this Matter leads me in some measure to
anticipate my Design, by taking notice of what seems to have had
a chief part in rendering Speculation intricate and perplexed,
and to have occasioned innumerable Errors and Difficulties in
almost all parts of Knowledge.  And that is the opinion that the
Mind hath a power of framing \emph{Abstract Ideas} or Notions
of Things.  He who is not a perfect Stranger to the Writings and
Disputes of Philosophers, must needs acknowledge that no small
part of them are spent about abstract Ideas.  These are in a more
especial manner, thought to be the Object of those Sciences which
go by the name of \emph{Logic} and \emph{Metaphysics}, and of
all that which passes under the Notion of the most abstracted and
sublime Learning, in all which one shall scarce find any Question
handled in such a manner, as does not suppose their Existence in
the Mind, and that it is well acquainted with them.



\paragraph{7.} It is agreed on all hands, that the Qualities or Modes of things
do never really exist each of them apart by it self, and
separated from all others, but are mix'd, as it were, and blended
together, several in the same Object.  But we are told, the Mind
being able to consider each Quality singly, or abstracted from
those other Qualities with which it is united, does by that means
frame to it self abstract Ideas.  For example, there is perceived
by Sight an Object extended, coloured, and moved: This mix'd or
compound Idea the mind resolving into its Simple, constituent
Parts, and viewing each by it self, exclusive of the rest, does
frame the abstract Ideas of Extension, Colour, and Motion.  Not
that it is possible for Colour or Motion to exist without
Extension: but only that the Mind can frame to it self by
\emph{Abstraction} the Idea of Colour exclusive of Extension,
and of Motion exclusive of both Colour and Extension.



\paragraph{8.} Again, the Mind having observed that in the particular Extensions
perceiv'd by Sense, there is something common and alike in all,
and some other things peculiar, as this or that Figure or
Magnitude, which distinguish them one from another; it considers
apart or singles out by it self that which is common, making
thereof a most abstract Idea of Extension, which is neither Line,
Surface, nor Solid, nor has any Figure or Magnitude but is an
Idea intirely prescinded from all these.  So likewise the Mind by
leaving out of the particular Colours perceived by Sense, that
which distinguishes them one from another, and retaining that
only which is common to all, makes an Idea of Colour in abstract
which is neither Red, nor Blue, nor White, nor any other
determinate Colour.  And in like manner by considering Motion
abstractedly not only from the Body moved, but likewise from the
Figure it describes, and all particular Directions and
Velocities, the abstract Idea of Motion is framed; which equally
corresponds to all particular Motions whatsoever that may be
perceived by Sense.



\paragraph{9.} And as the Mind frames to it self abstract Ideas of Qualities or
Modes, so does it, by the same precision or mental Separation,
attain abstract Ideas of the more compounded Beings, which
include several coexistent Qualities.  For example, the Mind
having observed that \emph{Peter}, \emph{James}, and
\emph{John}, resemble each other, in certain common Agreements
of Shape and other Qualities, leaves out of the complex or
compounded Idea it has of \emph{Peter}, \emph{James}, and any
other particular Man, that which is peculiar to each, retaining
only what is common to all; and so makes an abstract Idea wherein
all the particulars equally partake, abstracting intirely from
and cutting off all those Circumstances and Differences, which
might determine it to any particular Existence.  And after this
manner it is said we come by the abstract Idea of \emph{Man}
or, if you please, Humanity, or Humane Nature; wherein it is true
there is included Colour, because there is no Man but has some
Colour, but then it can be neither White, nor Black, nor any
particular Colour; because there is no one particular Colour
wherein all Men partake.  So likewise there is included Stature,
but then it is neither Tall Stature nor Low Stature, nor yet
Middle Stature, but something abstracted from all these.  And so
of the rest.  Moreover, there being a great variety of other
Creatures that partake in some Parts, but not all, of the complex
Idea of \emph{Man}, the Mind leaving out those Parts which are
peculiar to Men, and retaining those only which are common to all
the living Creatures, frameth the Idea of \emph{Animal}, which
abstracts not only from all particular Men, but also all Birds,
Beasts, Fishes, and Insects.  The constituent Parts of the
abstract Idea of Animal are Body, Life, Sense, and Spontaneous
Motion.  By \emph{Body} is meant, Body without any particular
Shape or Figure, there being no one Shape or Figure common to all
Animals, without Covering, either of Hair, or Feathers, or
Scales, \emph{\&c}.\ nor yet Naked: Hair, Feathers, Scales,
and Nakedness being the distinguishing Properties of particular
Animals, and for that reason left out of the \emph{Abstract
Idea}.  Upon the same account the spontaneous Motion must be
neither Walking, nor Flying, nor Creeping, it is nevertheless a
Motion, but what that Motion is, it is not easy to conceive.



\paragraph{10.} Whether others have this wonderful Faculty of \emph{Abstracting
their Ideas}, they best can tell: For my self I find indeed
I have a Faculty of imagining, or representing to myself the
Ideas of those particular things I have perceived and of
variously compounding and dividing them.  I can imagine a Man
with Two Heads or the upper parts of a Man joined to the Body of
a Horse.  I can consider the Hand, the Eye, the Nose, each by it
self abstracted or separated from the rest of the Body.  But then
whatever Hand or Eye I imagine, it must have some particular
Shape and Colour.  Likewise the Idea of Man that I frame to my
self, must be either of a White, or a Black, or a Tawny, a
Straight, or a Crooked, a Tall, or a Low, or a Middle-sized Man.
I cannot by any effort of Thought conceive the abstract Idea
above described.  And it is equally impossible for me to form the
abstract Idea of Motion distinct from the Body moving, and which
is neither Swift nor Slow, Curvilinear nor Rectilinear; and the
like may be said of all other abstract general Ideas whatsoever.
To be plain, I own my self able to abstract in one Sense, as when
I consider some particular Parts or Qualities separated from
others, with which though they are united in some Object, yet, it
is possible they may really Exist without them.  But I deny that
I can abstract one from another, or conceive separately, those
Qualities which it is impossible should Exist so separated; or
that I can frame a General Notion by abstracting from
Particulars in the manner aforesaid.  Which two last are the
proper Acceptations of \emph{Abstraction}.  And there are
Grounds to think most Men will acknowledge themselves to be in my
Case.  The Generality of Men which are Simple and Illiterate
never pretend to \emph{abstract Notions}.  It is said they are
difficult and not to be attained without Pains and Study.  We may
therefore reasonably conclude that, if such there be, they are
confined only to the Learned.



\paragraph{11.} I proceed to examine what can be alledged in defence of the
Doctrine of Abstraction, and try if I can discover what it is
that inclines the Men of Speculation to embrace an Opinion, so
remote from common Sense as that seems to be.  There has been a
late deservedly Esteemed Philosopher, who, no doubt, has given it
very much Countenance by seeming to think the having abstract
general Ideas is what puts the widest difference in point of
Understanding betwixt Man and Beast. ``The having of general
Ideas'' (\emph{saith he}) ``is that which puts a perfect
distinction betwixt Man and Brutes, and is an Excellency which
the Faculties of Brutes do by no means attain unto.  For it is
evident we observe no Footsteps in them of making use of general
Signs for universal Ideas; from which we have reason to imagine
that they have not the Faculty of \emph{abstracting} or making
general Ideas, since they have no use of Words or any other
general Signs.'' \emph{And a little after.} ``Therefore, I
think, we may suppose that it is in this that the Species of
Brutes are discriminated from Men, and 'tis that proper
difference wherein they are wholly separated, and which at last
widens to so wide a Distance.  For if they have any Ideas at
all, and are not bare Machines (as some would have them) we
cannot deny them to have some Reason.  It seems as evident to me
that they do some of them in certain Instances reason as that
they have Sense, but it is only in particular Ideas, just as they
receive them from their Senses.  They are the best of them tied
up within those narrow Bounds, and have not (as I think) the
Faculty to enlarge them by any kind of \emph{Abstraction}.''
\emph{Essay on Hum.  Underst}.\ B.~2.\ C.~11.\ Sect.\ 10 and 11.
I readily agree with this Learned Author, that the Faculties of
Brutes can by no means attain to \emph{Abstraction}.  But then
if this be made the distinguishing property of that sort of
Animals, I fear a great many of those that pass for Men must be
reckoned into their number.  The reason that is here assigned why
we have no Grounds to think Brutes have Abstract general Ideas,
is that we observe in them no use of Words or any other general
Signs; which is built on this Supposition, to wit, that the
making use of Words, implies the having general Ideas.  From
which it follows, that Men who use Language are able to Abstract
or Generalize their Ideas.  That this is the Sense and Arguing of
the Author will further appear by his answering the Question he
in another place puts.  ``Since all things that exist are only
Particulars, how come we by general Terms?''  \emph{His Answer
is}, ``Words become general by being made the Signs of
general Ideas.''
\emph{Essay on Hum. Underst}.\ B.~3.\ C.~3 Sect.~6.
But it seems that a Word becomes general by being made the Sign,
not of an abstract general Idea but, of several particular Ideas,
any one of which it indifferently suggests to the Mind.  For
Example, When it is said \emph{the change of Motion is proportional
to the impressed force}, or that \emph{whatever has Extension
is divisible}; these Propositions are to be understood of
Motion and Extension in general, and nevertheless it will not
follow that they suggest to my Thoughts an Idea of Motion without
a Body moved, or any determinate Direction and Velocity, or that
I must conceive an abstract general Idea of Extension, which is
neither Line, Surface nor Solid, neither Great nor Small, Black,
White, nor Red, nor of any other determinate Colour.  It is only
implied that whatever Motion I consider, whether it be Swift or
Slow, Perpendicular, Horizontal or Oblique, or in whatever
Object, the Axiom concerning it holds equally true.  As does the
other of every particular Extension, it matters not whether Line,
Surface or Solid, whether of this or that Magnitude or Figure.



\paragraph{12.} By observing how Ideas become general, we may the better judge
how Words are made so.  And here it is to be noted that I do not
deny absolutely there are general Ideas, but only that there are
any \emph{abstract general Ideas}: For in the Passages above
quoted, wherein there is mention of general Ideas, it is always
supposed that they are formed by \emph{Abstraction}, after the
manner set forth in
\emph{Sect.}~VIII and IX.
Now if we will annex a meaning to our Words, and speak only of
what we can conceive, I believe we shall acknowledge, that an
Idea, which considered in it self is particular, becomes general,
by being made to represent or stand for all other particular
Ideas of the same sort.  To make this plain by an Example,
suppose a Geometrician is demonstrating the Method, of cutting a
Line in two equal Parts.  He draws, for Instance, a Black Line of
an Inch in Length, this which in it self is a particular Line is
nevertheless with regard to its signification General, since as
it is there used, it represents all particular Lines whatsoever;
so that what is demonstrated of it, is demonstrated of all Lines,
or, in other Words, of a Line in General.  And as that particular
Line becomes General, by being made a Sign, so the name
\emph{Line} which taken absolutely is particular, by being a
Sign is made General.  And as the former owes its Generality, not
to its being the Sign of an abstract or general Line, but of all
particular right Lines that may possibly exist, so the latter
must be thought to derive its Generality from the same Cause,
namely, the various particular Lines which it indifferently
denotes.



\paragraph{13.} To give the Reader a yet clearer View of the Nature of abstract
Ideas, and the Uses they are thought necessary to, I shall add
one more Passage out of the \emph{Essay on Human
Understanding}, which is as follows.  ``\emph{Abstract
Ideas} are not so obvious or easy to Children or the yet
unexercised Mind as particular ones.  If they seem so to grown
Men, it is only because by constant and familiar Use they are
made so.  For when we nicely reflect upon them, we shall find
that general Ideas are Fictions and Contrivances of the Mind,
that carry Difficulty with them, and do not so easily offer
themselves, as we are apt to imagine.  For Example, Does it not
require some Pains and Skill to form the general Idea of a
Triangle (which is yet none of the most abstract, comprehensive
and difficult) for it must be neither Oblique nor Rectangle,
neither Equilateral, Equicrural, nor Scalenon, but \emph{all and
none} of these at once?  In effect, it is something imperfect
that cannot exist, an Idea wherein some Parts of several
different and \emph{inconsistent} Ideas are put together.  It
is true the Mind in this imperfect State has need of such Ideas,
and makes all the haste to them it can, for the conveniency of
Communication and Enlargement of Knowledge, to both which it is
naturally very much inclined.  But yet one has reason to suspect
such Ideas are Marks of our Imperfection.  At least this is
enough to shew that the most abstract and general Ideas are not
those that the Mind is first and most easily acquainted with, nor
such as its earliest Knowledge is conversant about.''
B.~4. C.~7. Sect.~9.
If any Man has the Faculty of framing in his Mind such an Idea of
a Triangle as is here described, it is in vain to pretend to
dispute him out of it, nor would I go about it.  All I desire is,
that the Reader would fully and certainly inform himself whether
he has such an Idea or no.  And this, methinks, can be no hard
Task for anyone to perform.  What more easy than for anyone to
look a little into his own Thoughts, and there try whether he
has, or can attain to have, an Idea that shall correspond with
the description that is here given of the general Idea of a
Triangle, which is, \emph{neither Oblique, nor Rectangle,
Equilateral, Equicrural, nor Scalenon, but all and none of these
at once?}



\paragraph{14.} Much is here said of the Difficulty that abstract Ideas carry
with them, and the Pains and Skill requisite to the forming them.
And it is on all Hands agreed that there is need of great Toil
and Labour of the Mind, to emancipate our Thoughts from
particular Objects, and raise them to those sublime Speculations
that are conversant about abstract Ideas.  From all which the
natural Consequence should seem to be, that so difficult a thing
as the forming abstract Ideas was not necessary for
\emph{Communication}, which is so easy and familiar to all
sorts of Men.  But we are told, if they seem obvious and easy to
grown Men, \emph{It is only because by constant and familiar use
they are made so}.  Now I would fain know at what time it is,
Men are imployed in surmounting that Difficulty, and furnishing
themselves with those necessary helps for Discourse.  It cannot
be when they are grown up, for then it seems they are not
conscious of any such Pains-taking; it remains therefore to be
the business of their Childhood.  And surely, the great and
multiplied Labour of framing abstract Notions, will be found a
hard Task for that tender Age.  Is it not a hard thing to
imagine, that a couple of Children cannot prate together, of
their Sugar-plumbs and Rattles and the rest of their little
Trinkets, till they have first tacked together numberless
Inconsistencies, and so framed in their Minds \emph{abstract
general Ideas}, and annexed them to every common Name they
make use of?



\paragraph{15.} Nor do I think them a whit more needful for the \emph{Enlargement
of Knowledge} than for \emph{Communication}.  It is I know
a Point much insisted on, that all Knowledge and Demonstration
are about universal Notions, to which I fully agree: But then it
doth not appear to me that those Notions are formed by
\emph{Abstraction} in the manner premised;
\emph{Universality}, so far as I can comprehend, not consisting
in the absolute, positive Nature or Conception of any thing, but
in the relation it bears to the Particulars signified or
represented by it: By virtue whereof it is that Things, Names, or
Notions, being in their own Nature \emph{Particular}, are
rendered \emph{Universal}.  Thus when I demonstrate any
Proposition concerning Triangles, it is to be supposed that I
have in view the universal Idea of a Triangle; which ought not to
be understood as if I could frame an Idea of a Triangle which was
neither Equilateral nor Scalenon nor Equicrural.  But only that
the particular Triangle I consider, whether of this or that sort
it matters not, doth equally stand for and represent all
Rectilinear Triangles whatsoever, and is in that sense
\emph{Universal}.  All which seems very Plain and not to
include any Difficulty in it.



\paragraph{16.} But here it will be demanded, how we can know any Proposition to
be true of all particular Triangles, except we have first seen it
demonstrated of the abstract Idea of a Triangle which equally
agrees to all? For because a Property may be demonstrated to
agree to some one particular Triangle, it will not thence follow
that it equally belongs to any other Triangle, which in all
respects is not the same with it.  For Example, Having
demonstrated that the three Angles of an Isosceles Rectangular
Triangle are equal to two right Ones, I cannot therefore conclude
this Affection agrees to all other Triangles, which have neither
a right Angle, nor two equal Sides.  It seems therefore that, to
be certain this Proposition is universally true, we must either
make a particular Demonstration for every particular Triangle,
which is impossible, or once for all demonstrate it of the
\emph{abstract Idea of a Triangle}, in which all the
Particulars do indifferently partake, and by which they are all
equally represented.  To which I answer, that though the Idea I
have in view whilst I make the Demonstration, be, for instance,
that of an Isosceles Rectangular Triangle, whose Sides are of a
determinate Length, I may nevertheless be certain it extends to
all other Rectilinear Triangles, of what Sort or Bigness soever.
And that, because neither the right Angle, nor the Equality, nor
determinate Length of the Sides, are at all concerned in the
Demonstration.  It is true, the Diagram I have in view includes
all these Particulars, but then there is not the least mention
made of them in the Proof of the Proposition.  It is not said,
the three Angles are equal to two right Ones, because one of them
is a right Angle, or because the Sides comprehending it are of
the same Length.  Which sufficiently shews that the right Angle
might have been Oblique, and the Sides unequal, and for all that
the Demonstration have held good.  And for this reason it is,
that I conclude that to be true of any Obliquangular or Scalenon,
which I had demonstrated of a particular Right-angled, Equicrural
Triangle; and not because I demonstrated the Proposition of the
abstract Idea of a Triangle.  And here it must be acknowledged
that a Man may consider a Figure merely as triangular, without
attending to the particular Qualities of the Angles, or relations
of the Sides.  So far he may abstract: But this will never prove,
that he can frame an abstract general inconsistent Idea of a
Triangle.  In like manner we may consider \emph{Peter} so far
forth as Man, or so far forth as Animal, without framing the
forementioned abstract Idea, either of Man or of Animal, in as
much as all that is perceived is not considered.



\paragraph{17.} It were an endless, as well as an useless Thing, to trace the
\emph{Schoolmen}, those great Masters of Abstraction, through
all the manifold inextricable Labyrinths of Error and Dispute,
which their Doctrine of abstract Natures and Notions seems to
have led them into.  What Bickerings and Controversies, and what
a learned Dust have been raised about those Matters, and what
mighty Advantage hath been from thence derived to Mankind, are
things at this Day too clearly known to need being insisted on.
And it had been well if the ill Effects of that Doctrine were
confined to those only who make the most avowed Profession of it.
When Men consider the great Pains, Industry and Parts, that have
for so many Ages been laid out on the Cultivation and Advancement
of the Sciences, and that notwithstanding all this, the far
greater Part of them remain full of Darkness and Uncertainty, and
Disputes that are like never to have an end, and even those that
are thought to be supported by the most clear and cogent
Demonstrations, contain in them Paradoxes which are perfectly
irreconcilable to the Understandings of Men, and that taking all
together, a small Portion of them doth supply any real Benefit to
Mankind, otherwise than by being an innocent Diversion and
Amusement: I say, the Consideration of all this is apt to throw
them into a Despondency, and perfect Contempt of all Study.  But
this may perhaps cease, upon a view of the false Principles that
have obtained in the World, amongst all which there is none,
methinks, hath a more wide Influence over the Thoughts of
Speculative Men, than this of abstract general Ideas.



\paragraph{18.} I come now to consider the Source of this prevailing Notion, and
that seems to me to be Language.  And surely nothing of less
extent than Reason it self could have been the Source of an
Opinion so universally received.  The truth of this appears as
from other Reasons, so also from the plain Confession of the
ablest Patrons of abstract Ideas, who acknowledge that they are
made in order to naming; from which it is a clear Consequence,
that if there had been no such thing as Speech or Universal
Signs, there never had been any thought of Abstraction.
\emph{See} B.~3. C.~6. Sect.~39.
\emph{and elsewhere of the Essay on Human Understanding.} Let
us therefore examine the manner wherein Words have contributed to
the Origin of that Mistake.  First then, 'Tis thought that every
Name hath, or ought to have, one only precise and settled
Signification, which inclines Men to think there are certain
\emph{abstract, determinate Ideas}, which constitute the true
and only immediate Signification of each general Name.  And that
it is by the mediation of these abstract Ideas, that a general
Name comes to signify any particular Thing.  Whereas, in truth,
there is no such thing as one precise and definite Signification
annexed to any general Name, they all signifying indifferently a
great number of particular Ideas.  All which doth evidently
follow from what has been already said, and will clearly appear
to anyone by a little Reflexion.  To this it will be objected,
that every Name that has a Definition, is thereby restrained to
one certain Signification.  For Example, a \emph{Triangle} is
defined to be \emph{a plain Surface comprehended by three right
Lines}; by which that Name is limited to denote one certain
Idea and no other.  To which I answer, that in the Definition it
is not said whether the Surface be Great or Small, Black or
White, nor whether the Sides are Long or Short, Equal or Unequal,
nor with what Angles they are inclined to each other; in all
which there may be great Variety, and consequently there is no
one settled Idea which limits the Signification of the word
\emph{Triangle}.  'Tis one thing for to keep a Name constantly
to the same Definition, and another to make it stand every where
for the same Idea: the one is necessary, the other useless and
impracticable.



\paragraph{19.} But to give a farther Account how Words came to produce the
Doctrine of abstract Ideas, it must be observed that it is a
received Opinion, that Language has no other End but the
communicating our Ideas, and that every significant Name stands
for an Idea.  This being so, and it being withal certain, that
Names, which yet are not thought altogether insignificant, do not
always mark out particular conceivable Ideas, it is straightway
concluded that they stand for abstract Notions.  That there are
many Names in use amongst Speculative Men, which do not always
suggest to others determinate particular Ideas, is what no Body
will deny.  And a little Attention will discover, that it is not
necessary (even in the strictest Reasonings) significant Names
which stand for Ideas should, every time they are used, excite in
the Understanding the Ideas they are made to stand for:  In
Reading and Discoursing, Names being for the most part used as
Letters are in \emph{Algebra}, in which though a particular
quantity be marked by each Letter, yet to proceed right it is not
requisite that in every step each Letter suggest to your
Thoughts, that particular quantity it was appointed to stand for.



\paragraph{20.} Besides, the communicating of Ideas marked by Words is not the
chief and only end of Language, as is commonly supposed.  There
are other Ends, as the raising of some Passion, the exciting to,
or deterring from an Action, the putting the Mind in some
particular Disposition; to which the former is in many Cases
barely subservient, and sometimes intirely omitted, when these
can be obtained without it, as I think doth not unfrequently
happen in the familiar use of Language.  I intreat the Reader to
reflect with himself, and see if it doth not often happen either
in Hearing or Reading a Discourse, that the Passions of Fear,
Love, Hatred, Admiration, Disdain, and the like, arise
immediately in his Mind upon the Perception of certain Words,
without any Ideas coming between.  At first, indeed, the Words
might have occasioned Ideas that were fit to produce those
Emotions; but, if I mistake not, it will be found that when
Language is once grown familiar, the hearing of the Sounds or
Sight of the Characters is oft immediately attended with those
Passions, which at first were wont to be produced by the
intervention of Ideas, that are now quite omitted.  May we not,
for Example, be affected with the promise of a \emph{good
Thing}, though we have not an Idea of what it is? Or is not
the being threatned with Danger sufficient to excite a Dread,
though we think not of any particular Evil likely to befal us,
nor yet frame to our selves an Idea of Danger in Abstract? If any
one shall join ever so little Reflexion of his own to what has
been said, I believe it will evidently appear to him, that
general Names are often used in the propriety of Language without
the Speaker's designing them for Marks of Ideas in his own, which
he would have them raise in the Mind of the Hearer.  Even proper
Names themselves do not seem always spoken, with a Design to
bring into our view the Ideas of those Individuals that are
supposed to be marked by them.  For Example, when a Schoolman
tells me \emph{Aristotle hath said it}, all I conceive he means
by it, is to dispose me to embrace his Opinion with the Deference
and Submission which Custom has annexed to that Name.  And this
effect may be so instantly produced in the Minds of those who are
accustomed to resign their Judgment to the Authority of that
Philosopher, as it is impossible any Idea either of his Person,
Writings, or Reputation should go before.  Innumerable Examples
of this kind may be given, but why should I insist on those
things, which every one's Experience will, I doubt not,
plentifully suggest unto him?



\paragraph{21.} We have, I think, shewn the Impossibility of \emph{abstract
Ideas}.  We have considered what has been said for them by
their ablest Patrons; and endeavored to shew they are of no Use
for those Ends, to which they are thought necessary.  And lastly,
we have traced them to the Source from whence they flow, which
appears to be Language.  It cannot be denied that Words are of
excellent Use, in that by their means all that Stock of Knowledge
which has been purchased by the joint Labours of inquisitive Men
in all Ages and Nations, may be drawn into the view and made the
possession of one single Person.  But at the same time it must be
owned that most parts of Knowledge have been strangely perplexed
and darkened by the abuse of Words, and general ways of Speech
wherein they are delivered.  Since therefore Words are so apt to
impose on the Understanding, whatever Ideas I consider, I shall
endeavour to take them bare and naked into my View, keeping out
of my Thoughts, so far as I am able, those Names which long and
constant Use hath so strictly united with them; from which I may
expect to derive the following Advantages.



\paragraph{22.} First, I shall be sure to get clear of all Controversies purely
Verbal; the springing up of which Weeds in almost all the
Sciences has been a main Hindrance to the Growth of true and
sound Knowledge.  Secondly, this seems to be a sure way to
extricate my self out of that fine and subtile Net of
\emph{abstract Ideas}, which has so miserably perplexed and
entangled the Minds of Men, and that with this peculiar
Circumstance, that by how much the finer and more curious was the
Wit of any Man, by so much the deeper was he like to be
ensnared, and faster held therein. Thirdly, so long as I confine
my Thoughts to my own Ideas divested of Words, I do not see how I
can easily be mistaken.  The Objects I consider, I clearly and
adequately know.  I cannot be deceived in thinking I have an Idea
which I have not.  It is not possible for me to imagine, that any
of my own Ideas are alike or unlike, that are not truly so.  To
discern the Agreements or Disagreements there are between my
Ideas, to see what Ideas are included in any compound Idea, and
what not, there is nothing more requisite, than an attentive
Perception of what passes in my own Understanding.



\paragraph{23.} But the attainment of all these Advantages doth presuppose an
intire Deliverance from the Deception of Words, which I dare
hardly promise my self; so difficult a thing it is to dissolve an
Union so early begun, and confirmed by so long a Habit as that
betwixt Words and Ideas.  Which Difficulty seems to have been
very much increased by the Doctrine of \emph{Abstraction}.  For
so long as Men thought abstract Ideas were annexed to their
Words, it doth not seem strange that they should use Words for
Ideas: It being found an impracticable thing to lay aside the
Word, and retain the abstract Idea in the Mind, which in it self
was perfectly inconceivable.  This seems to me the principal
Cause, why those Men who have so emphatically recommended to
others, the laying aside all use of Words in their Meditations,
and contemplating their bare Ideas, have yet failed to perform it
themselves.  Of late many have been very sensible of the absurd
Opinions and insignificant Disputes, which grow out of the abuse
of Words.  And in order to remedy these Evils they advise well,
that we attend to the Ideas signified, and draw off our Attention
from the Words which signify them.  But how good soever this
Advice may be, they have given others, it is plain they could not
have a due regard to it themselves, so long as they thought the
only immediate use of Words was to signify Ideas, and that the
immediate Signification of every general Name was a
\emph{determinate, abstract Idea}.



\paragraph{24.} But these being known to be Mistakes, a Man may with greater Ease
prevent his being imposed on by Words.  He that knows he has no
other than particular Ideas, will not puzzle himself in vain to
find out and conceive the abstract Idea, annexed to any Name.
And he that knows Names do not always stand for Ideas, will spare
himself the labour of looking for Ideas, where there are none to
be had.  It were therefore to be wished that every one would use
his utmost Endeavours, to obtain a clear View of the Ideas he
would consider, separating from them all that dress and
incumbrance of Words which so much contribute to blind the
Judgment and divide the Attention.  In vain do we extend our View
into the Heavens, and pry into the Entrails of the Earth, in vain
do we consult the Writings of learned Men, and trace the dark
Footsteps of Antiquity; we need only draw the Curtain of Words,
to behold the fairest Tree of Knowledge, whose Fruit is
excellent, and within the reach of our Hand.



\paragraph{25.} Unless we take care to clear the first Principles of Knowledge,
from the embarras and delusion of Words, we may make infinite
Reasonings upon them to no purpose; we may draw Consequences from
Consequences, and be never the wiser.  The farther we go, we
shall only lose our selves the more irrecoverably, and be the
deeper entangled in Difficulties and Mistakes.  Whoever therefore
designs to read the following Sheets, I intreat him to make my
Words the Occasion of his own Thinking, and endeavour to attain
the same Train of Thoughts in Reading, that I had in writing
them.  By this means it will be easy for him to discover the
Truth or Falsity of what I say.  He will be out of all danger of
being deceived by my Words, and I do not see how he can be led
into an Error by considering his own naked, undisguised Ideas.

\end{sectionbody}

\newpage

\itemsection{Part I}

\begin{sectionbody}

\paragraph{1.} It is evident to any one who takes a Survey of the Objects of
Humane Knowledge, that they are either Ideas actually imprinted
on the Senses, or else such as are perceived by attending to the
Passions and Operations of the Mind, or lastly Ideas formed by
help of Memory and Imagination, either compounding, dividing, or
barely representing those originally perceived in the aforesaid
ways.  By Sight I have the Ideas of Light and Colours with their
several Degrees and Variations.  By Touch I perceive, for
Example, Hard and Soft, Heat and Cold, Motion and Resistance, and
of all these more and less either as to Quantity or Degree.
Smelling furnishes me with Odors; the Palate with Tastes, and
Hearing conveys Sounds to the Mind in all their variety of Tone
and Composition.  And as several of these are observed to
accompany each other, they come to be marked by one Name, and so
to be reputed as one Thing.  Thus, for Example, a certain Colour,
Taste, Smell, Figure and Consistence having been observed to go
together, are accounted one distinct Thing, signified by the Name
\emph{Apple}.  Other collections of Ideas constitute a Stone, a
Tree, a Book, and the like sensible Things; which, as they are
pleasing or disagreeable, excite the Passions of Love, Hatred,
Joy, Grief, and so forth.



\paragraph{2.} But besides all that endless variety of Ideas or Objects of
Knowledge, there is likewise something which knows or perceives
them, and exercises divers Operations, as Willing, Imagining,
Remembering about them.  This perceiving, active Being is what I
call \emph{Mind}, \emph{Spirit}, \emph{Soul} or \emph{my
Self}.  By which Words I do not denote any one of my Ideas,
but a thing intirely distinct from them, wherein they exist, or,
which is the same thing, whereby they are perceived; for the
Existence of an Idea consists in being perceived.



\paragraph{3.} That neither our Thoughts, nor Passions, nor Ideas formed by the
Imagination, exist without the Mind, is what every Body will
allow.  And it seems no less evident that the various Sensations
or Ideas imprinted on the Sense, however blended or combined
together (that is, whatever Objects they compose) cannot exist
otherwise than in a Mind perceiving them.  I think an intuitive
Knowledge may be obtained of this, by any one that shall attend
to what is meant by the Term \emph{Exist} when applied to
sensible Things.  The Table I write on, I say, exists, that is, I
see and feel it; and if I were out of my Study I should say it
existed, meaning thereby that if I was in my Study I might
perceive it, or that some other Spirit actually does perceive it.
There was an Odor, that is, it was smelled; There was a Sound,
that is to say, it was heard; a Colour or Figure, and it was
perceived by Sight or Touch.  This is all that I can understand
by these and the like Expressions.  For as to what is said of the
absolute Existence of unthinking Things without any relation to
their being perceived, that seems perfectly unintelligible.
Their \emph{Esse} is \emph{Percipi}, nor is it possible they
should have any Existence, out of the Minds or thinking Things
which perceive them.



\paragraph{4.} It is indeed an Opinion strangely prevailing amongst Men, that
Houses, Mountains, Rivers, and in a word all sensible Objects
have an Existence Natural or Real, distinct from their being
perceived by the Understanding.  But with how great an Assurance
and Acquiescence soever this Principle may be entertained in the
World; yet whoever shall find in his Heart to call it in
Question, may, if I mistake not, perceive it to involve a
manifest Contradiction.  For what are the forementioned Objects
but the things we perceive by Sense, and what do we perceive
besides our own Ideas or Sensations; and is it not plainly
repugnant that any one of these or any Combination of them should
exist unperceived?



\paragraph{5.} If we thoroughly examine this Tenet, it will, perhaps, be found
at Bottom to depend on the Doctrine of \emph{Abstract Ideas}.  For
can there be a nicer Strain of Abstraction than to distinguish
the Existence of sensible Objects from their being perceived, so
as to conceive them Existing unperceived? Light and Colours, Heat
and Cold, Extension and Figures, in a word the Things we see and
feel, what are they but so many Sensations, Notions, Ideas or
Impressions on the Sense; and is it possible to separate, even in
thought, any of these from Perception? For my part I might as
easily divide a Thing from it Self.  I may indeed divide in my
Thoughts or conceive apart from each other those Things which,
perhaps, I never perceived by Sense so divided.  Thus I imagine
the Trunk of a Humane Body without the Limbs, or conceive the
Smell of a Rose without thinking on the Rose it self.  So far I
will not deny I can abstract, if that may properly be called
\emph{Abstraction}, which extends only to the conceiving
separately such Objects, as it is possible may really exist or be
actually perceived asunder.  But my conceiving or imagining Power
does not extend beyond the possibility of real Existence or
Perception.  Hence as it is impossible for me to see or feel any
Thing without an actual Sensation of that Thing, so is it
impossible for me to conceive in my Thoughts any sensible Thing
or Object distinct from the Sensation or Perception of it.



\paragraph{6.} Some Truths there are so near and obvious to the Mind, that a Man
need only open his Eyes to see them.  Such I take this Important
one to be, to wit, that all the Choir of Heaven and Furniture of
the Earth, in a word all those Bodies which compose the mighty
Frame of the World, have not any Subsistence without a Mind, that
their Being is to be perceived or known; that consequently so
long as they are not actually perceived by me, or do not exist in
my Mind or that of any other created Spirit, they must either
have no Existence at all, or else subsist in the Mind of some
eternal Spirit: It being perfectly unintelligible and involving
all the Absurdity of Abstraction, to attribute to any single part
of them an Existence independent of a Spirit.  To be convinced of
which, the Reader need only reflect and try to separate in his
own Thoughts the being of a sensible thing from its being
perceived.



\paragraph{7.} From what has been said, it follows, there is not any other
Substance than \emph{Spirit}, or that which perceives.  But for
the fuller proof of this Point, let it be considered, the
sensible Qualities are Colour, Figure, Motion, Smell, Taste, and
such like, that is, the Ideas perceived by Sense.  Now for an
Idea to exist in an unperceiving Thing, is a manifest
Contradiction; for to have an Idea is all one as to perceive:
that therefore wherein Colour, Figure, and the like Qualities
exist, must perceive them; hence it is clear there can be no
unthinking Substance or \emph{Substratum} of those Ideas.



\paragraph{8.} But say you, though the Ideas themselves do not exist without the
Mind, yet there may be Things like them whereof they are Copies
or Resemblances, which Things exist without the Mind, in an
unthinking Substance.  I answer, an Idea can be like nothing but
an Idea; a Colour or Figure can be like nothing but another
Colour or Figure.  If we look but ever so little into our
Thoughts, we shall find it impossible for us to conceive a
Likeness except only between our Ideas.  Again, I ask whether
those supposed Originals or external Things, of which our Ideas
are the Pictures or Representations, be themselves perceivable or
no?  If they are, then they are Ideas, and we have gained our
Point; but if you say they are not, I appeal to any one whether
it be Sense, to assert a Colour is like something which is
invisible; Hard or Soft, like something which is Intangible; and
so of the rest.



\paragraph{9.} Some there are who make a Distinction betwixt \emph{Primary}
and \emph{Secondary} Qualities: By the former, they mean
Extension, Figure, Motion, Rest, Solidity or Impenetrability and
Number: By the latter they denote all other sensible Qualities,
as Colours, Sounds, Tastes, and so forth.  The Ideas we have of
these they acknowledge not to be the Resemblances of any thing
existing without the Mind or unperceived; but they will have our
Ideas of the primary Qualities to be Patterns or Images of Things
which exist without the Mind, in an unthinking Substance which
they call \emph{Matter}.  By Matter therefore we are to
understand an inert, senseless Substance, in which Extension,
Figure, and Motion, do actually subsist.  But it is evident from
what we have already shewn, that Extension, Figure and Motion are
only Ideas existing in the Mind, and that an Idea can be like
nothing but another Idea, and that consequently neither They nor
their Archetypes can exist in an unperceiving Substance.  Hence
it is plain, that that the very Notion of what is called
\emph{Matter} or \emph{Corporeal Substance}, involves a
Contradiction in it.



\paragraph{10.} They who assert that Figure, Motion, and the rest of the Primary
or Original Qualities do exist without the Mind, in unthinking
Substances, do at the same time acknowledge that Colours, Sounds,
Heat, Cold, and suchlike secondary Qualities, do not, which they
tell us are Sensations existing in the Mind alone, that depend on
and are occasioned by the different Size, Texture and Motion of
the minute Particles of Matter.  This they take for an undoubted
Truth, which they can demonstrate beyond all Exception.  Now if
it be certain, that those original Qualities are inseparably
united with the other sensible Qualities, and not, even in
Thought, capable of being abstracted from them, it plainly
follows that they exist only in the Mind.  But I desire any one
to reflect and try, whether he can by any Abstraction of Thought,
conceive the Extension and Motion of a Body, without all other
sensible Qualities.  For my own part, I see evidently that it is
not in my power to frame an Idea of a Body extended and moved,
but I must withal give it some Colour or other sensible Quality
which is acknowledged to exist only in the Mind.  In short,
Extension, Figure, and Motion, abstracted from all other
Qualities, are inconceivable.  Where therefore the other sensible
Qualities are, there must these be also, to wit, in the Mind and
no where else.



\paragraph{11.} Again, \emph{Great} and \emph{Small}, \emph{Swift} and
\emph{Slow}, are allowed to exist no where without the Mind,
being intirely relative, and changing as the Frame or Position of
the Organs of Sense varies.  The Extension therefore which exists
without the Mind, is neither great nor small, the Motion neither
swift nor slow, that is, they are nothing at all.  But, say you,
they are Extension in general, and Motion in general: Thus we see
how much the Tenet of extended, moveable Substances existing
without the Mind, depends on that strange Doctrine of
\emph{abstract Ideas}.  And here I cannot but remark, how
nearly the Vague and indeterminate Description of Matter or
corporeal Substance, which the Modern Philosophers are run into
by their own Principles, resembles that antiquated and so much
ridiculed Notion of \emph{Materia prima}, to be met with in
\emph{Aristotle} and his Followers.  Without Extension Solidity
cannot be conceived; since therefore it has been shewn that
Extension exists not in an unthinking Substance, the same must
also be true of Solidity.



\paragraph{12.} That Number is intirely the Creature of the Mind, even though the
other Qualities be allowed to exist without, will be evident to
whoever considers, that the same thing bears a different
Denomination of Number, as the Mind views it with different
respects.  Thus, the same Extension is One or Three or Thirty
Six, according as the Mind considers it with reference to a Yard,
a Foot, or an Inch.  Number is so visibly relative, and dependent
on Mens Understanding, that it is strange to think how any one
should give it an absolute Existence without the Mind.  We say
one Book, one Page, one Line; all these are equally Unites,
though some contain several of the others.  And in each Instance
it is plain, the Unite relates to some particular Combination of
Ideas arbitrarily put together by the Mind.



\paragraph{13.} Unity I know some will have to be a simple or uncompounded Idea,
accompanying all other Ideas into the Mind.  That I have any such
Idea answering the Word \emph{Unity}, I do not find; and if I
had, methinks I could not miss finding it; on the contrary it
should be the most familiar to my Understanding, since it is said
to accompany all other Ideas, and to be perceived by all the ways
of Sensation and Reflexion.  To say no more, it is an
\emph{abstract Idea}.



\paragraph{14.} I shall farther add, that after the same manner, as modern
Philosophers prove certain sensible Qualities to have no
Existence in Matter, or without the Mind, the same thing may be
likewise proved of all other sensible Qualities whatsoever.
Thus, for Instance, it is said that Heat and Cold are Affections
only of the Mind, and not at all Patterns of real Beings,
existing in the corporeal Substances which excite them, for that
the same Body which appears Cold to one Hand, seems Warm to
another.  Now why may we not as well argue that Figure and
Extension are not Patterns or Resemblances of Qualities existing
in Matter, because to the same Eye at different Stations, or Eyes
of a different Texture at the same Station, they appear various,
and cannot therefore be the Images of any thing settled and
determinate without the Mind?  Again, It is proved that Sweetness
is not really in the sapid Thing, because the thing remaining
unaltered the Sweetness is changed into Bitter, as in case of a
Fever or otherwise vitiated Palate.  Is it not as reasonable to
say, that Motion is not without the Mind, since if the Succession
of Ideas in the Mind become swifter, the Motion, it is
acknowledged, shall appear slower without any Alteration in any
external Object?



\paragraph{15.} In short, let any one consider those Arguments, which are thought
manifestly to prove that Colours and Tastes exist only in the
Mind, and he shall find they may with equal force, be brought to
prove the same thing of Extension, Figure, and Motion.  Though it
must be confessed this Method of arguing doth not so much prove
that there is no Extension or Colour in an outward Object, as
that we do not know by Sense which is the true Extension or
Colour of the Object.  But the Arguments foregoing plainly shew
it to be impossible that any Colour or Extension at all, or other
sensible Quality whatsoever, should exist in an unthinking
Subject without the Mind, or in truth, that there should be any
such thing as an outward Object.



\paragraph{16.} But let us examine a little the received Opinion.  It is said
Extension is a Mode or Accident of Matter, and that Matter is the
\emph{Substratum} that supports it.  Now I desire that you
would explain what is meant by Matter's \emph{supporting}
Extension: Say you, I have no Idea of Matter, and therefore
cannot explain it.  I answer, though you have no positive, yet if
you have any meaning at all, you must at least have a relative
Idea of Matter; though you know not what it is, yet you must be
supposed to know what Relation it bears to Accidents, and what is
meant by its supporting them.  It is evident \emph{Support}
cannot here be taken in its usual or literal Sense, as when we
say that Pillars support a Building: In what Sense therefore must
it be taken?



\paragraph{17.} If we inquire into what the most accurate Philosophers declare
themselves to mean by \emph{Material Substance}; we shall find
them acknowledge, they have no other meaning annexed to those
Sounds, but the Idea of Being in general, together with the
relative Notion of its supporting Accidents.  The general Idea of
Being appeareth to me the most abstract and incomprehensible of
all other; and as for its supporting Accidents, this, as we have
just now observed, cannot be understood in the common Sense of
those Words; it must therefore be taken in some other Sense, but
what that is they do not explain.  So that when I consider the
two Parts or Branches which make the signification of the Words
\emph{Material Substance}, I am convinced there is no distinct
meaning annexed to them.  But why should we trouble our selves
any farther, in discussing this Material \emph{Substratum} or
Support of Figure and Motion, and other sensible Qualities? Does
it not suppose they have an Existence without the Mind? And is
not this a direct Repugnancy, and altogether inconceivable?



\paragraph{18.} But though it were possible that solid, figured, moveable
Substances may exist without the Mind, corresponding to the Ideas
we have of Bodies, yet how is it possible for us to know this?
Either we must know it by Sense, or by Reason.  As for our
Senses, by them we have the Knowledge only of our Sensations,
Ideas, or those things that are immediately perceived by Sense,
call them what you will: But they do not inform us that things
exist without the Mind, or unperceived, like to those which are
perceived.  This the Materialists themselves acknowledge.  It
remains therefore that if we have any Knowledge at all of
external Things, it must be by Reason, inferring their Existence
from what is immediately perceived by Sense.  But what reason can
induce us to believe the Existence of Bodies without the Mind,
from what we perceive, since the very Patrons of Matter
themselves do not pretend, there is any necessary Connexion
betwixt them and our Ideas? I say it is granted on all hands (and
what happens in Dreams, Phrensies, and the like, puts it beyond
dispute) that it is possible we might be affected with all the
Ideas we have now, though no Bodies existed without, resembling
them.  Hence it is evident the Supposition of external Bodies is
not necessary for the producing our Ideas: Since it is granted
they are produced sometimes, and might possibly be produced
always in the same Order we see them in at present, without their
Concurrence.



\paragraph{19.} But though we might possibly have all our Sensations without
them, yet perhaps it may be thought easier to conceive and
explain the manner of their Production, by supposing external
Bodies in their likeness rather than otherwise; and so it might
be at least probable there are such things as Bodies that excite
their Ideas in our Minds.  But neither can this be said; for
though we give the Materialists their external Bodies, they by
their own confession are never the nearer knowing how our Ideas
are produced: Since they own themselves unable to comprehend in
what manner Body can act upon Spirit, or how it is possible it
should imprint any Idea in the Mind.  Hence it is evident the
Production of Ideas or Sensations in our Minds, can be no reason
why we should suppose Matter or corporeal Substances, since that
is acknowledged to remain equally inexplicable with, or without
this Supposition.  If therefore it were possible for Bodies to
exist without the Mind, yet to hold they do so, must needs be a
very precarious Opinion; since it is to suppose, without any
reason at all, that God has created innumerable Beings that are
intirely useless, and serve to no manner of purpose.



\paragraph{20.} In short, if there were external Bodies, it is impossible we
should ever come to know it; and if there were not, we might have
the very same Reasons to think there were that we have now.
Suppose, what no one can deny possible, an Intelligence, without
the help of external Bodies, to be affected with the same train
of Sensations or Ideas that you are, imprinted in the same order
and with like vividness in his Mind.  I ask whether that
Intelligence hath not all the Reason to believe the Existence of
corporeal Substances, represented by his Ideas, and exciting them
in his Mind, that you can possibly have for believing the same
thing? Of this there can be no Question; which one Consideration
is enough to make any reasonable Person suspect the strength of
whatever Arguments be may think himself to have, for the
Existence of Bodies without the Mind.



\paragraph{21.} Were it necessary to add any farther Proof against the Existence
of Matter, after what has been said, I could instance several of
those Errors and Difficulties (not to mention Impieties) which
have sprung from that Tenet.  It has occasioned numberless
Controversies and Disputes in Philosophy, and not a few of far
greater moment in Religion.  But I shall not enter into the
detail of them in this Place, as well because I think, Arguments
\emph{\`{a} Posteriori} are unnecessary for confirming what
has been, if I mistake not, sufficiently demonstrated
\emph{\`{a} Priori}, as because I shall hereafter find
occasion to say somewhat of them.



\paragraph{22.} I am afraid I have given cause to think me needlesly prolix in
handling this Subject.  For to what purpose is it to dilate on
that which may be demonstrated with the utmost Evidence in a Line
or two, to any one that is capable of the least Reflexion? It is
but looking into your own Thoughts, and so trying whether you can
conceive it possible for a Sound, or Figure, or Motion, or
Colour, to exist without the Mind, or unperceived.  This easy
Trial may make you see, that what you contend for, is a downright
Contradiction.  Insomuch that I am content to put the whole upon
this Issue; if you can but conceive it possible for one extended
moveable Substance, or in general, for any one Idea or any thing
like an Idea, to exist otherwise than in a Mind perceiving it, I
shall readily give up the Cause: And as for all that
\emph{compages} of external Bodies which you contend for, I
shall grant you its Existence, though you cannot either give me
any Reason why you believe it exists, or assign any use to it
when it is supposed to exist.  I say, the bare possibility of
your Opinion's being true, shall pass for an Argument that it is
so.



\paragraph{23.} But say you, surely there is nothing easier than to imagine
Trees, for instance, in a Park, or Books existing in a Closet,
and no Body by to perceive them.  I answer, you may so, there is
no difficulty in it: But what is all this, I beseech you, more
than framing in your Mind certain Ideas which you call
\emph{Books} and \emph{Trees}, and the same time omitting to
frame the Idea of any one that may perceive them? But do not you
your self perceive or think of them all the while?  This
therefore is nothing to the purpose: It only shews you have the
Power of imagining or forming Ideas in your Mind; but it doth not
shew that you can conceive it possible, the Objects of your
Thought may exist without the Mind: To make out this, it is
necessary that you conceive them existing unconceived or
unthought of, which is a manifest Repugnancy.  When we do our
utmost to conceive the Existence of external Bodies, we are all
the while only contemplating our own Ideas.  But the Mind taking
no notice of it self, is deluded to think it can and doth
conceive Bodies existing unthought of or without the Mind; though
at the same time they are apprehended by or exist in it self.  A
little Attention will discover to any one the Truth and Evidence
of what is here said, and make it unnecessary to insist on any
other Proofs against the Existence of material Substance.



\paragraph{24.} It is very obvious, upon the least Inquiry into our own Thoughts,
to know whether it be possible for us to understand what is
meant, by the \emph{absolute Existence of sensible Objects in
themselves, or without the Mind}.  To me it is evident those
Words mark out either a direct Contradiction, or else nothing at
all.  And to convince others of this, I know no readier or fairer
way, than to intreat they would calmly attend to their own
Thoughts: And if by this Attention, the Emptiness or Repugnancy
of those Expressions does appear, surely nothing more is
requisite for their Conviction.  It is on this therefore that I
insist, to wit, that the absolute Existence of unthinking Things
are Words without a Meaning, or which include a Contradiction.
This is what I repeat and inculcate, and earnestly recommend to
the attentive Thoughts of the Reader.



\paragraph{25.} All our Ideas, Sensations, or the things which we perceive, by
whatsoever Names they may be distinguished, are visibly inactive,
there is nothing of Power or Agency included in them.  So that
one Idea or Object of Thought cannot produce, or make any
Alteration in another.  To be satisfied of the Truth of this,
there is nothing else requisite but a bare Observation of our
Ideas.  For since they and every part of them exist only in the
Mind, it follows that there is nothing in them but what is
perceived.  But whoever shall attend to his Ideas, whether of
Sense or Reflexion, will not perceive in them any Power or
Activity; there is therefore no such thing contained in them.
A little Attention will discover to us that the very Being of an
Idea implies Passiveness and Inertness in it, insomuch that it is
impossible for an Idea to do any thing, or, strictly speaking, to
be the Cause of any thing: Neither can it be the Resemblance or
Pattern of any active Being, as is evident from \emph{Sect.}~8.
Whence it plainly follows that Extension, Figure and Motion,
cannot be the Cause of our Sensations.  To say therefore, that
these are the effects of Powers resulting from the Configuration,
Number, Motion, and Size of Corpuscles, must certainly be false.



\paragraph{26.} We perceive a continual Succession of Ideas, some are anew
excited, others are changed or totally disappear.  There is
therefore some Cause of these Ideas whereon they depend, and
which produces and changes them.  That this Cause cannot be any
Quality or Idea or Combination of Ideas, is clear from the
preceding Section.  It must therefore be a Substance; but it has
been shewn that there is no corporeal or material Substance: It
remains therefore that the Cause of Ideas is an incorporeal
active Substance or Spirit.



\paragraph{27.} A Spirit is one simple, undivided, active Being: as it perceives
Ideas, it is called the \emph{Understanding}, and as it
produces or otherwise operates about them, it is called the
\emph{Will}.  Hence there can be no Idea formed of a
Soul or Spirit: For all Ideas whatever, being Passive and Inert,
\emph{vide Sect.}~25.\ they cannot represent unto us, by way of
Image or Likeness, that
which acts.  A little Attention will make it plain to any one,
that to have an Idea which shall be like that active Principle of
Motion and Change of Ideas, is absolutely impossible.  Such is
the Nature of \emph{Spirit} or that which acts, that it cannot
be of it self perceived, but only by the Effects which it
produceth.  If any Man shall doubt of the Truth of what is here
delivered, let him but reflect and try if he can frame the Idea
of any Power or active Being; and whether he hath Ideas of two
principal Powers, marked by the Names \emph{Will} and
\emph{Understanding}, distinct from each other as well as from
a third Idea of Substance or Being in general, with a relative
Notion of its supporting or being the Subject of the aforesaid
Powers, which is signified by the Name \emph{Soul} or
\emph{Spirit}.  This is what some hold; but so far as I can
see, the Words \emph{Will}, \emph{Soul}, \emph{Spirit}, do
not stand for different Ideas, or in truth, for any Idea at all,
but for something which is very different from Ideas, and which
being an Agent cannot be like unto, or represented by, any Idea
whatsoever.  Though it must be owned at the same time, that we
have some Notion of Soul, Spirit, and the Operations of the Mind,
such as Willing, Loving, Hating, in as much as we know or
understand the meaning of those Words.



\paragraph{28.} I find I can excite Ideas in my Mind at pleasure, and vary and
shift the Scene as oft as I think fit.  It is no more than
Willing, and straightway this or that Idea arises in my Fancy:
And by the same Power it is obliterated, and makes way for
another.  This making and unmaking of Ideas doth very properly
denominate the Mind active.  Thus much is certain, and grounded
on Experience: But when we think of unthinking Agents, or of
exciting Ideas exclusive of Volition, we only amuse our selves
with Words.



\paragraph{29.} But whatever Power I may have over my own Thoughts, I find the
Ideas actually perceived by Sense have not a like Dependence on
my Will.  When in broad Day-light I open my Eyes, it is not in my
Power to choose whether I shall see or no, or to determine what
particular Objects shall present themselves to my View; and so
likewise as to the Hearing and other Senses, the Ideas imprinted
on them are not Creatures of my Will.  There is therefore some
other Will or Spirit that produces them.



\paragraph{30.} The Ideas of Sense are more strong, lively, and distinct than
those of the Imagination; they have likewise a Steddiness, Order,
and Coherence, and are not excited at random, as those which are
the effects of Humane Wills often are, but in a regular Train or
Series, the admirable Connexion whereof sufficiently testifies
the Wisdom and Benevolence of its Author.  Now the set Rules or
established Methods, wherein the Mind we depend on excites in us
the Ideas of Sense, are called the \emph{Laws of Nature}: And
these we learn by Experience, which teaches us that such and such
Ideas are attended with such and such other Ideas, in the
ordinary course of Things.



\paragraph{31.} This gives us a sort of Foresight, which enables us to regulate
our Actions for the benefit of Life.  And without this we should
be eternally at a loss: We could not know how to act any thing
that might procure us the least Pleasure, or remove the least
Pain of Sense.  That Food nourishes, Sleep refreshes, and Fire
warms us; that to sow in the Seed-time is the way to reap in the
Harvest, and, in general, that to obtain such or such Ends, such
or such Means are conducive, all this we know, not by discovering
any necessary Connexion between our Ideas, but only by the
Observation of the settled Laws of Nature, without which we
should be all in Uncertainty and Confusion, and a grown Man no
more know how to manage himself in the Affairs of Life, than an
Infant just born.



\paragraph{32.} And yet this consistent uniform working, which so evidently
displays the Goodness and Wisdom of that governing Spirit whose
Will constitutes the Laws of Nature, is so far from leading our
Thoughts to him, that it rather sends them a wandering after
second Causes.  For when we perceive certain Ideas of Sense
constantly followed by other Ideas, and we know this is not of
our own doing, we forthwith attribute Power and Agency to the
Ideas themselves, and make one the Cause of another, than which
nothing can be more absurd and unintelligible.  Thus, for
Example, having observed that when we perceive by Sight a certain
round luminous Figure, we at the same time perceive by Touch the
Idea or Sensation called \emph{Heat}, we do from thence
conclude the Sun to be the cause of Heat.  And in like manner
perceiving the Motion and Collision of Bodies to be attended with
Sound, we are inclined to think the latter an effect of the
former.



\paragraph{33.} The Ideas imprinted on the Senses by the Author of Nature are
called \emph{real Things}: And those excited in the Imagination
being less regular, vivid and constant, are more properly termed
\emph{Ideas}, or \emph{Images of Things}, which they copy and
represent.  But then our Sensations, be they never so vivid and
distinct, are nevertheless \emph{Ideas}, that is, they exist in
the Mind, or are perceived by it, as truly as the Ideas of its
own framing.  The Ideas of Sense are allowed to have more reality
in them, that is, to be more strong, orderly, and coherent than
the Creatures of the Mind; but this is no Argument that they
exist without the Mind.  They are also less dependent on the
Spirit, or thinking Substance which perceives them, in that they
are excited by the Will of another and more powerful Spirit: yet
still they are \emph{Ideas}, and certainly no \emph{Idea},
whether faint or strong, can exist otherwise than in a Mind
perceiving it.



\paragraph{34.} Before we proceed any farther, it is necessary to spend some Time
in answering Objections which may probably be made against the
Principles hitherto laid down.  In doing of which, if I seem too
prolix to those of quick Apprehensions, I hope it may be
pardoned, since all Men do not equally apprehend things of this
Nature; and I am willing to be understood by every one.  First
then, it will be objected that by the foregoing Principles, all
that is real and substantial in Nature is banished out of the
World: And instead thereof a chimerical Scheme of Ideas takes
place.  All things that exist, exist only in the Mind, that is,
they are purely notional.  What therefore becomes of the Sun,
Moon, and Stars?  What must we think of Houses, Rivers,
Mountains, Trees, Stones; nay, even of our own Bodies? Are all
these but so many Chimeras and Illusions on the Fancy? To all
which, and whatever else of the same sort may be objected, I
answer, that by the Principles premised, we are not deprived of
any one thing in Nature.  Whatever we see, feel, hear, or any
wise conceive or understand, remains as secure as ever, and is as
real as ever.  There is a \emph{rerum natura} \[nature of things\], and the
Distinction between Realities and Chimeras retains its full
force.  This is evident from
\emph{Sect.}\ 29, 30, and 33,
where we have shewn what is meant by
\emph{real Things} in opposition to \emph{Chimeras}, or Ideas
of our own framing; but then they both equally exist in the Mind,
and in that Sense they are alike \emph{Ideas}.



\paragraph{35.} I do not argue against the Existence of any one thing that we can
apprehend, either by Sense or Reflexion.  That the things I see
with mine Eyes and touch with my Hands do exist, really exist, I
make not the least Question.  The only thing whose Existence we
deny, is that which \emph{Philosophers} call Matter or
corporeal Substance.  And in doing of this, there is no Damage
done to the rest of Mankind, who, I dare say, will never miss it.
The Atheist indeed will want the Colour of an empty Name to
support his Impiety; and the Philosophers may possibly find, they
have lost a great Handle for Trifling and Disputation.



\paragraph{36.} If any Man thinks this detracts from the Existence or Reality of
Things, he is very far from understanding what hath been premised
in the plainest Terms I could think of.  Take here an Abstract of
what has been said.  There are spiritual Substances, Minds, or
humane Souls, which will or excite Ideas in themselves at
pleasure: but these are faint, weak, and unsteady in respect of
others they perceive by Sense,  which being impressed upon them
according to certain Rules or Laws of Nature, speak themselves
the Effects of a Mind more powerful and wise than humane Spirits.
These latter are said to have more \emph{Reality} in them than
the former: By which is meant that they are more affecting,
orderly, and distinct, and that they are not Fictions of the Mind
perceiving them.  And in this Sense, the Sun that I see by Day is
the real Sun, and that which I imagine by Night is the Idea of
the former.  In the Sense here given of \emph{Reality}, it is
evident that every Vegetable, Star, Mineral, and in general each
part of the Mundane System, is as much a \emph{real Being} by
our Principles as by any other.  Whether others mean any thing by
the Term \emph{Reality} different from what I do, I intreat
them to look into their own Thoughts and see.



\paragraph{37.} It will be urged that thus much at least is true, to wit, that we
take away all corporeal Substances.  To this my Answer is, That
if the word \emph{Substance} be taken in the vulgar Sense, for
a Combination of sensible Qualities, such as Extension, Solidity,
Weight, and the like; This we cannot be accused of taking away.
But if it be taken in a philosophic Sense, for the support of
Accidents or Qualities without the Mind: Then indeed I
acknowledge that we take it away, if one may be said to take away
that which never had any Existence, not even in the Imagination.



\paragraph{38.} But, say you, it sounds very harsh to say we eat and drink Ideas,
and are clothed with Ideas.  I acknowledge it does so, the word
\emph{Idea} not being used in common Discourse to signify the
several Combinations of sensible Qualities, which are called
\emph{Things}: and it is certain that any Expression which
varies from the familiar Use of Language, will seem harsh and
ridiculous.  But this doth not concern the Truth of the
Proposition, which in other Words is no more than to say, we are
fed and clothed with those Things which we perceive immediately
by our Senses.  The Hardness or Softness, the Colour, Taste,
Warmth, Figure, and such like Qualities, which combined together
constitute the several sorts of Victuals and Apparel, have been
shewn to exist only in the Mind that perceives them; and this is
all that is meant by calling them \emph{Ideas}; which Word, if
it was as ordinarily used as \emph{Thing}, would sound no
harsher nor more ridiculous than it.  I am not for disputing
about the Propriety, but the Truth of the Expression.  If
therefore you agree with me that we eat and drink, and are clad
with the immediate Objects of Sense which cannot exist
unperceived or without the Mind: I shall readily grant it is more
proper or conformable to Custom, that they should be called
Things rather than Ideas.



\paragraph{39.} If it be demanded why I make use of the word \emph{Idea}, and
do not rather in compliance with Custom call them
\emph{Things}, I answer, I do it for two Reasons: First,
because the Term \emph{Thing}, in contradistinction to
\emph{Idea}, is generally supposed to denote somewhat existing
without the Mind: Secondly, because \emph{Thing} hath a more
comprehensive Signification than \emph{Idea}, including Spirits
or thinking Things as well as Ideas.  Since therefore the Objects
of Sense exist only in the Mind, and are withal thoughtless and
inactive, I chose to mark them by the word \emph{Idea}, which
implies those Properties.



\paragraph{40.} But say what we can, some one perhaps may be apt to reply, he
will still believe his Senses, and never suffer any Arguments,
how plausible soever, to prevail over the Certainty of them.  Be
it so, assert the Evidence of Sense as high as you please, we are
willing to do the same.  That what I see, hear and feel doth
exist, that is to say, is perceived by me, I no more doubt than I
do of my own Being.  But I do not see how the Testimony of Sense
can be alledged, as a proof for the Existence of any thing, which
is not perceived by Sense.  We are not for having any Man turn
\emph{Sceptic}, and disbelieve his Senses; on the contrary we
give them all the Stress and Assurance imaginable; nor are there
any Principles more opposite to Scepticism, than those we have
laid down, as shall be hereafter clearly shewn.



\paragraph{41.} Secondly, it will be objected that there is a great difference
betwixt real Fire, for Instance, and the Idea of Fire, betwixt
dreaming or imagining ones self burnt, and actually being so:
This and the like may be urged in opposition to our Tenets.  To
all which the Answer is evident from what hath been already said,
and I shall only add in this place, that if real Fire be very
different from the Idea of Fire, so also is the real Pain that it
occasions, very different from the Idea of the same Pain: and yet
no Body will pretend that real Pain either is, or can possibly
be, in an unperceiving Thing or without the Mind, any more than
its Idea.



\paragraph{42.} Thirdly, It will be objected that we see Things actually without
or at distance from us, and which consequently do not exist in
the Mind, it being absurd that those Things which are seen at the
distance of several Miles, should be as near to us as our own
Thoughts.  In answer to this, I desire it may be considered, that
in a Dream we do oft perceive Things as existing at a great
distance off, and yet for all that, those Things are acknowledged
to have their Existence only in the Mind.



\paragraph{43.} But for the fuller clearing of this Point, it may be worth while
to consider, how it is that we perceive Distance and Things
placed at a Distance by Sight.  For that we should in truth see
external Space, and Bodies actually existing in it, some nearer,
others farther off, seems to carry with it some Opposition to
what hath been said, of their existing no where without the Mind.
The Consideration of this Difficulty it was, that gave birth to
my \emph{Essay towards a new Theory of Vision}, which was
published not long since.  Wherein it is shewn that
\emph{Distance} or Outness is neither immediately of it self
perceived by Sight, nor yet apprehended or judged of by Lines and
Angles, or any thing that hath a necessary Connexion with it: But
that it is only suggested to our Thoughts, by certain visible
Ideas and Sensations attending Vision, which in their own Nature
have no manner of Similitude or Relation, either with Distance,
or Things placed at a Distance.  But by a Connexion taught us by
Experience, they come to signify and suggest them to us, after
the same manner that Words of any Language suggest the Ideas they
are made to stand for.  Insomuch that a Man born blind, and
afterwards made to see, would not, at first Sight, think the
Things he saw, to be without his Mind, or at any Distance from
him.  See \emph{Sect.}~41.\ of the forementioned Treatise.



\paragraph{44.} The Ideas of Sight and Touch make two Species, intirely distinct
and heterogeneous.  The former are Marks and Prognostics of the
latter.  That the proper Objects of Sight neither exist without
the Mind, nor are the Images of external Things, was shewn even
in that Treatise.  Though throughout the same, the contrary be
supposed true of tangible Objects: Not that to suppose that
vulgar Error, was necessary for establishing the Notion therein
laid down; but because it was beside my Purpose to examine and
refute it in a Discourse concerning \emph{Vision}.  So that in
strict Truth the Ideas of Sight, when we apprehend by them
Distance and Things placed at a Distance, do not suggest or mark
out to us Things actually existing at a Distance, but only
admonish us what Ideas of Touch will be imprinted in our Minds at
such and such distances of Time, and in consequence of such or
such Actions.  It is, I say, evident from what has been said in
the foregoing Parts of this Treatise, and in \emph{Sect.}~147,
and elsewhere of the Essay concerning Vision, that visible Ideas
are the Language whereby the governing Spirit, on whom we depend,
informs us what tangible Ideas he is about to imprint upon us, in
case we excite this or that Motion in our own Bodies.  But for a
fuller Information in this Point, I refer to the Essay it self.



\paragraph{45.} Fourthly, It will be objected that from the foregoing Principles
it follows, Things are every moment annihilated and created anew.
The Objects of Sense exist only when they are perceived: The
Trees therefore are in the Garden, or the Chairs in the Parlour,
no longer than while there is some body by to perceive them.
Upon shutting my Eyes all the Furniture in the Room is reduced to
nothing, and barely upon opening them it is again created.  In
answer to all which, I refer the Reader to what has been said in
\emph{Sect.}\ 3, 4, \emph{\&c.}\ and desire he will consider
whether he means any thing by the
actual Existence of an Idea, distinct from its being perceived.
For my part, after the nicest Inquiry I could make, I am not able
to discover that any thing else is meant by those Words.  And I
once more intreat the Reader to sound his own Thoughts, and not
suffer himself to be imposed on by Words.  If he can conceive it
possible either for his Ideas or their Archetypes to exist
without being perceived, then I give up the Cause: But if he
cannot, he will acknowledge it is unreasonable for him to stand
up in defence of he knows not what, and pretend to charge on me
as an Absurdity, the not assenting to those Propositions which at
Bottom have no meaning in them.



\paragraph{46.} It will not be amiss to observe, how far the received Principles
of Philosophy are themselves chargeable with those pretended
Absurdities.  It is thought strangely absurd that upon closing my
Eyelids, all the visible Objects round me should be reduced to
nothing; and yet is not this what Philosophers commonly
acknowledge, when they agree on all hands, that Light and
Colours, which alone are the proper and immediate Objects of
Sight, are mere Sensations that exist no longer than they are
perceived?  Again, it may to some perhaps seem very incredible,
that things should be every moment creating, yet this very Notion
is commonly taught in the Schools.  For the \emph{Schoolmen},
though they acknowledge the Existence of Matter, and that the
whole mundane Fabrick is framed out of it, are nevertheless of
Opinion that it cannot subsist without the Divine Conservation,
which by them is expounded to be a continual Creation.



\paragraph{47.} Farther, a little Thought will discover to us, that though we
allow the Existence of Matter or Corporeal Substance, yet it will
unavoidably follow from the Principles which are now generally
admitted, that the particular Bodies of what kind soever, do none
of them exist whilst they are not perceived.  For it is evident
from \emph{Sect.}~11.  and the following Sections, that the
Matter Philosophers contend for, is an incomprehensible Somewhat
which hath none of those particular Qualities, whereby the Bodies
falling under our Senses are distinguished one from another.  But
to make this more plain, it must be remarked, that the infinite
Divisibility of Matter is now universally allowed, at least by
the most approved and considerable Philosophers, who on the
received Principles demonstrate it beyond all Exception.  Hence
it follows, that there is an infinite Number of Parts in each
Particle of Matter, which are not perceived by Sense.  The Reason
therefore, that any particular Body seems to be of a finite
Magnitude, or exhibits only a finite Number of Parts to Sense,
is, not because it contains no more, since in itself it contains
an infinite Number of Parts, but because the Sense is not acute
enough to discern them.  In proportion therefore as the Sense is
rendered more acute, it perceives a greater Number of Parts in
the Object, that is, the Object appears greater, and its Figure
varies, those Parts in its Extremities which were before
unperceivable, appearing now to bound it in very different Lines
and Angles from those perceived by an obtuser Sense.  And at
length, after various Changes of Size and Shape, when the Sense
becomes infinitely acute, the Body shall seem infinite.  During
all which there is no Alteration in the Body, but only in the
Sense.  Each Body therefore considered in it self, is infinitely
extended, and consequently void of all Shape or Figure.  From
which it follows, that though we should grant the Existence of
Matter to be ever so certain, yet it is withal as certain, the
Materialists themselves are by their own Principles forced to
acknowledge, that neither the particular Bodies perceived by
Sense, nor any thing like them exists without the Mind.  Matter,
I say, and each Particle thereof is according to them infinite
and shapeless, and it is the Mind that frames all that variety of
Bodies which compose the visible World, any one whereof does not
exist longer than it is perceived.



\paragraph{48.} If we consider it, the Objection proposed in
\emph{Sect.}~45.\ will not be found reasonably charged on the
Principles we have
premised, so as in truth to make any Objection at all against our
Notions.  For though we hold indeed the Objects of Sense to be
nothing else but Ideas which cannot exist unperceived; yet we may
not hence conclude they have no Existence except only while they
are perceived by us, since there may be some other Spirit that
perceives them, though we do not.  Wherever Bodies are said to
have no Existence without the Mind, I would not be understood to
mean this or that particular Mind, but all Minds whatsoever.  It
does not therefore follow from the foregoing Principles, that
Bodies are annihilated and created every moment, or exist not at
all during the Intervals between our Perception of them.



\paragraph{49.} Fifthly, It may perhaps be objected, that if Extension and Figure
exist only in the Mind, it follows that the Mind is extended and
figured; since Extension is a Mode or Attribute, which (to speak
with the Schools) is predicated of the Subject in which it
exists.  I answer, Those Qualities are in the Mind only as they
are perceived by it, that is, not by way of \emph{Mode} or
\emph{Attribute}, but only by way of \emph{Idea}; and it no
more follows, that the Soul or Mind is extended because Extension
exists in it alone, than it does that it is red or blue, because
those Colours are on all hands acknowledged to exist in it, and
no where else.  As to what Philosophers say of Subject and Mode,
that seems very groundless and unintelligible.  For Instance, in
this Proposition, a Die is hard, extended, and square, they will
have it that the Word \emph{Die} denotes a Subject or
Substance, distinct from the Hardness, Extension and Figure,
which are predicated of it, and in which they exist.  This I
cannot comprehend: To me a Die seems to be nothing distinct from
those things which are termed its Modes or Accidents.  And to
say a Die is hard, extended and square, is not to attribute those
Qualities to a Subject distinct from and supporting them, but
only an Explication of the meaning of the Word \emph{Die}.



\paragraph{50.} Sixthly, You will say there have been a great many things
explained by Matter and Motion: Take away these, and you destroy
the whole Corpuscular Philosophy, and undermine those mechanical
Principles which have been applied with so much Success to
account for the \emph{Ph{\ae}nomena}.  In short, whatever
Advances have been made, either by ancient or modern
Philosophers, in the study of Nature, do all proceed on the
Supposition, that Corporeal Substance or Matter doth really
exist.  To this I answer, that there is not any one
\emph{Ph{\ae}nomenon} explained on that Supposition, which may
not as well be explained without it, as might easily be made
appear by an Induction of Particulars.  To explain the
\emph{Ph{\ae}nomena}, is all one as to shew, why upon such
and such Occasions we are affected with such and such Ideas.  But
how Matter should operate on a Spirit, or produce any Idea in it,
is what no Philosopher will pretend to explain.  It is therefore
evident, there can be no use of Matter in Natural Philosophy.
Besides, they who attempt to account for Things, do it not by
Corporeal Substance, but by Figure, Motion, and other Qualities,
which are in truth no more than mere Ideas, and therefore cannot
be the Cause of any thing, as hath been already shewn.
See \emph{Sect.}~25.



\paragraph{51.} Seventhly, It will upon this be demanded whether it does not seem
absurd to take away natural Causes, and ascribe every thing to
the immediate Operation of Spirits? We must no longer say upon
these Principles that Fire heats, or Water cools, but that a
Spirit heats, and so forth.  Would not a Man be deservedly laught
at, who should talk after this manner? I answer, he would so; in
such things we ought to \emph{think with the Learned, and speak
with the Vulgar}. They who to Demonstration are convinced of the
truth of the \emph{Copernican} System, do nevertheless say the
Sun rises, the Sun sets, or comes to the Meridian: And if they
affected a contrary Stile in common talk, it would without doubt
appear very ridiculous.  A little Reflexion on what is here said
will make it manifest, that the common use of Language would
receive no manner of Alteration or Disturbance from the Admission
of our Tenets.



\paragraph{52.} In the ordinary Affairs of Life, any Phrases may be retained, so
long as they excite in us proper Sentiments, or Dispositions to
act in such a manner as is necessary for our well-being, how
false soever they may be, if taken in a strict and speculative
Sense.  Nay this is unavoidable, since Propriety being regulated
by Custom, Language is suited to the received Opinions, which are
not always the truest.  Hence it is impossible, even in the most
rigid philosophic Reasonings, so far to alter the Bent and Genius
of the Tongue we speak, as never to give a handle for Cavillers
to pretend Difficulties and Inconsistencies.  But a fair and
ingenuous Reader will collect the Sense, from the Scope and Tenor
and Connexion of a Discourse, making allowances for those
inaccurate Modes of Speech, which use has made inevitable.



\paragraph{53.} As to the Opinion that there are no Corporeal Causes, this has
been heretofore maintained by some of the Schoolmen, as it is of
late by others among the modern Philosophers, who though they
allow Matter to exist, yet will have {\sc God} alone to be the
immediate efficient Cause of all things.  These Men saw, that
amongst all the Objects of Sense, there was none which had any
Power or Activity included in it, and that by Consequence this
was likewise true of whatever Bodies they supposed to exist
without the Mind, like unto the immediate Objects of Sense.  But
then, that they should suppose an innumerable Multitude of
created Beings, which they acknowledge are not capable of
producing any one Effect in Nature, and which therefore are made
to no manner of purpose, since God might have done every thing as
well without them; this I say, though we should allow it
possible, must yet be a very unaccountable and extravagant
Supposition.



\paragraph{54.} In the eighth place, The universal concurrent Assent of Mankind
may be thought by some, an invincible Argument in behalf of
Matter, or the Existence of external things.  Must we suppose the
whole World to be mistaken? And if so, what Cause can be assigned
of so widespread and predominant an Error? I answer, First, That
upon a narrow Inquiry, it will not perhaps be found, so many as
is imagined do really believe the Existence of Matter or Things
without the Mind.  Strictly speaking, to believe that which
involves a Contradiction, or has no meaning in it, is impossible:
And whether the foregoing Expressions are not of that sort, I
refer it to the impartial Examination of the Reader.  In one
sense indeed, Men may be said to believe that Matter exists, that
is, they act as if the immediate Cause of their Sensations, which
affects them every moment and is so nearly present to them, were
some senseless unthinking Being.  But that they should clearly
apprehend any Meaning marked by those Words, and form thereof a
settled speculative Opinion, is what I am not able to conceive.
This is not the only Instance wherein Men impose upon themselves,
by imagining they believe those Propositions they have often
heard, though at bottom they have no meaning in them.



\paragraph{55.} But secondly, Though we should grant a Notion to be ever so
universally and stedfastly adhered to, yet this is but a weak
Argument of its Truth, to whoever considers what a vast number of
Prejudices and false Opinions are every where embraced with the
utmost Tenaciousness, by the unreflecting (which are the far
greater) Part of Mankind.  There was a time when the
\emph{Antipodes} and Motion of the Earth were looked upon as
monstrous Absurdities, even by Men of Learning: And if it be
considered what a small proportion they bear to the rest of
Mankind, we shall find that at this Day, those Notions have
gained but a very inconsiderable footing in the World.



\paragraph{56.} But it is demanded, that we assign a Cause of this Prejudice, and
account for its obtaining in the World.  To this I answer, That
Men knowing they perceived several Ideas, whereof they themselves
were not the Authors, as not being excited from within, nor
depending on the Operation of their Wills, this made them
maintain, those Ideas or Objects of Perception had an Existence
independent of, and without the Mind, without ever dreaming that
a Contradiction was involved in those Words.  But Philosophers
having plainly seen, that the immediate Objects of Perception do
not exist without the Mind, they in some degree corrected the
mistake of the Vulgar, but at the same time run into another
which seems no less absurd, to wit, that there are certain
Objects really existing without the Mind, or having a Subsistence
distinct from being perceived, of which our Ideas are only Images
or Resemblances, imprinted by those Objects on the Mind.  And
this Notion of the Philosophers owes its Origin to the same Cause
with the former, namely, their being conscious that they were not
the Authors of their own Sensations, which they evidently knew
were imprinted from without, and which therefore must have some
Cause, distinct from the Minds on which they are imprinted.



\paragraph{57.} But why they should suppose the Ideas of Sense to be excited in
us by things in their likeness, and not rather have recourse to
\emph{Spirit} which alone can act, may be accounted for, First,
because they were not aware of the Repugnancy there is, as well
in supposing things like unto our Ideas existing without, as in
attributing to them Power or Activity.  Secondly, because the
supreme Spirit which excites those Ideas in our Minds, is not
marked out and limited to our view by any particular finite
Collection of sensible Ideas, as humane Agents are by their Size,
Complexion, Limbs, and Motions.  And thirdly, because his
Operations are regular and uniform.  Whenever the Course of
Nature is interrupted by a Miracle, Men are ready to own the
Presence of a superior Agent.  But when we see things go on in
the ordinary Course, they do not excite in us any Reflexion;
their Order and Concatenation, though it be an Argument of the
greatest Wisdom, Power, and Goodness in their Creator, is yet so
constant and familiar to us, that we do not think them the
immediate Effects of a \emph{Free Spirit}: especially since
Inconstancy and Mutability in acting, though it be an
Imperfection, is looked on as a mark of \emph{Freedom}.



\paragraph{58.} Tenthly, It will be objected, that the Notions we advance, are
inconsistent with several sound Truths in Philosophy and
Mathematicks.  For Example, The Motion of the Earth is now
universally admitted by Astronomers, as a Truth grounded on the
clearest and most convincing Reasons;  but on the foregoing
Principles, there can be no such thing.  For Motion being only an
Idea, it follows that if it be not perceived, it exists not; but
the Motion of the Earth is not perceived by Sense.  I answer,
That Tenet, if rightly understood, will be found to agree with
the Principles we have premised: For the Question, whether the
Earth moves or no, amounts in reality to no more than this, to
wit, whether we have reason to conclude from what hath been
observed by Astronomers, that if we were placed in such and such
Circumstances, and such or such a Position and Distance, both
from the Earth and Sun, we should perceive the former to move
among the Choir of the Planets, and appearing in all respects
like one of them: And this, by the established Rules of Nature,
which we have no reason to mistrust, is reasonably collected from
the Ph{\ae}nomena.



\paragraph{59.} We may, from the Experience we have had of the Train and
Succession of Ideas in our Minds, often make, I will not say
uncertain Conjectures, but sure and well-grounded Predictions,
concerning the Ideas we shall be affected with, pursuant to a
great Train of Actions, and be enabled to pass a right Judgment
of what would have appeared to us, in case we were placed in
Circumstances very different from those we are in at present.
Herein consists the Knowledge of Nature, which may preserve its
Use and Certainty very consistently with what hath been said.  It
will be easy to apply this to whatever Objections of the like
sort may be drawn from the Magnitude of the Stars, or any other
Discoveries in Astronomy or Nature.



\paragraph{60.} In the eleventh place, It will be demanded to what purpose serves
that curious Organization of Plants, and the admirable Mechanism
in the Parts of Animals; might not Vegetables grow, and shoot
forth Leaves and Blossoms, and Animals perform all their Motions,
as well without as with all that variety of internal Parts so
elegantly contrived and put together, which being Ideas have
nothing powerful or operative in them, nor have any necessary
Connexion with the Effects ascribed to them? If it be a Spirit
that immediately produces every Effect by a \emph{Fiat}, or Act
of his Will, we must think all that is fine and artificial in the
Works, whether of Man or Nature, to be made in vain.  By this
Doctrine, though an Artist hath made the Spring and Wheels, and
every Movement of a Watch, and adjusted them in such a manner as
he knew would produce the Motions he designed; yet he must think
all this done to no purpose, and that it is an Intelligence which
directs the Index, and points to the Hour of the Day.  If so, why
may not the Intelligence do it, without his being at the pains of
making the Movements, and putting them together? Why does not an
empty Case serve as well as another? And how comes it to pass,
that whenever there is any Fault in the going of a Watch, there
is some corresponding Disorder to be found in the Movements,
which being mended by a skilful Hand, all is right again?  The
like may be said of all the Clockwork of Nature, great part
whereof is so wonderfully fine and subtile, as scarce to be
discerned by the best Microscope.  In short, it will be asked,
how upon our Principles any tolerable Account can be given, or
any final Cause assigned of an innumerable multitude of Bodies
and Machines framed with the most exquisite Art, which in the
common Philosophy have very apposite uses assigned them, and
serve to explain abundance of Ph{\ae}nomena.



\paragraph{61.} To all which I answer, First, That though there were some
Difficulties relating to the Administration of Providence, and
the uses by it assigned to the several parts of Nature, which I
could not solve by the foregoing Principles, yet this Objection
could be of small weight against the Truth and Certainty of those
things which may be proved \emph{\`{a} priori}, with the
utmost Evidence.  Secondly, But neither are the received
Principles free from the like Difficulties; for it may still be
demanded, to what end God should take those round-about Methods
of effecting things by Instruments and Machines, which no one can
deny might have been effected by the mere Command of his Will,
without all that \emph{apparatus}: Nay, if we narrowly consider
it, we shall find the Objection may be retorted with greater
force on those who hold the Existence of those Machines without
the Mind; for it has been made evident, that Solidity, Bulk,
Figure, Motion and the like, have no \emph{Activity} or
\emph{Efficacy} in them, so as to be capable of producing any
one Effect in Nature.  See \emph{Sect.}~25.
Whoever therefore supposes them to exist (allowing the
Supposition possible) when they are not perceived, does it
manifestly to no purpose; since the only use that is assigned to
them, as they exist unperceived, is that they produce those
perceivable Effects, which in truth cannot be ascribed to any
thing but Spirit.



\paragraph{62.} But to come nearer the Difficulty, it must be observed, that
though the Fabrication of all those Parts and Organs be not
absolutely necessary to the producing any Effect, yet it is
necessary to the producing of things in a constant, regular way,
according to the Laws of Nature.  There are certain general Laws
that run through the whole Chain of natural Effects: These are
learned by the Observation and Study of Nature, and are by Men
applied as well to the framing artificial things for the Use and
Ornament of Life, as to the explaining the various
\emph{Ph{\ae}nomena}: Which Explication consists only in
shewing the Conformity any particular Ph{\ae}nomenon hath to
the general Laws of Nature, or, which is the same thing, in
discovering the \emph{Uniformity} there is in the Production of
natural Effects; as will be evident to whoever shall attend to
the several Instances, wherein Philosophers pretend to account
for Appearances.  That there is a great and conspicuous Use in
these regular constant Methods of working observed by the Supreme
Agent, hath been shewn in \emph{Sect.}~31.
And it is no less visible, that a particular Size, Figure, Motion
and Disposition of Parts are necessary, though not absolutely to
the producing any Effect, yet to the producing it according to
the standing mechanical Laws of Nature.  Thus, for Instance, it
cannot be denied that God, or the Intelligence which sustains and
rules the ordinary Course of things might, if He were minded to
produce a Miracle, cause all the Motions on the Dial-plate of a
Watch, though no Body had ever made the Movements, and put them
in it: But yet if he will act agreeably to the Rules of
Mechanism, by him for wise ends established and maintained in the
Creation, it is necessary that those Actions of the Watchmaker,
whereby he makes the Movements and rightly adjusts them, precede
the Production of the aforesaid Motions; as also that any
Disorder in them be attended with the Perception of some
corresponding Disorder in the Movements, which being once
corrected all is right again.



\paragraph{63.} It may indeed on some Occasions be necessary, that the Author of
Nature display his overruling Power in producing some Appearance
out of the ordinary Series of things.  Such Exceptions from the
general Rules of Nature are proper to surprise and awe Men into
an Acknowledgment of the Divine Being: But then they are to be
used but seldom, otherwise there is a plain Reason why they
should fail of that Effect.  Besides, God seems to choose the
convincing our Reason of his Attributes by the Works of Nature,
which discover so much Harmony and Contrivance in their Make, and
are such plain Indications of Wisdom and Beneficence in their
Author, rather than to astonish us into a belief of his Being by
anomalous and surprising Events.



\paragraph{64.} To set this Matter in a yet clearer Light, I shall observe that
what has been objected in \emph{Sect.}~60.\ amounts in reality to
no more than this: Ideas are not any how
and at random produced, there being a certain Order and Connexion
between them, like to that of Cause and Effect: There are also
several Combinations of them, made in a very regular and
artificial manner, which seem like so many Instruments in the
hand of Nature, that being hid as it were behind the Scenes, have
a secret Operation in producing those Appearances which are seen
on the Theatre of the World, being themselves discernible only to
the curious Eye of the Philosopher.  But since one Idea cannot be
the Cause of another, to what purpose is that Connexion? And
since those Instruments, being barely \emph{inefficacious
Perceptions} in the Mind, are not subservient to the
Production of natural Effects; it is demanded why they are made,
or, in other Words, what reason can be assigned why God should
make us, upon a close Inspection into his Works, behold so great
Variety of Ideas, so artfully laid together, and so much
according to Rule; it not being credible, that he would be at the
Expense (if one may so speak) of all that Art and Regularity to
no purpose?



\paragraph{65.} To all which my Answer is, First, That the Connexion of Ideas
does not imply the Relation of \emph{Cause} and
\emph{Effect}, but only of a Mark or \emph{Sign} with the
thing \emph{signified}.  The Fire which I see is not the Cause
of the Pain I suffer upon my approaching it, but the Mark that
forewarns me of it.  In like manner, the Noise that I hear is not
the Effect of this or that Motion or Collision of the ambient
Bodies, but the Sign thereof.  Secondly, The Reason why Ideas are
formed into Machines, that is, artificial and regular
Combinations, is the same with that for combining Letters into
Words.  That a few Original Ideas may be made to signify a great
number of Effects and Actions, it is necessary they be variously
combined together: And to the end their use be permanent and
universal, these Combinations must be made by \emph{Rule}, and
with \emph{wise Contrivance}.  By this means abundance of
Information is conveyed unto us, concerning what we are to expect
from such and such Actions, and what Methods are proper to be
taken, for the exciting such and such Ideas: Which in effect is
all that I conceive to be distinctly meant, when it is said that
by discerning the Figure, Texture, and Mechanism of the inward
Parts of Bodies, whether natural or artificial, we may attain to
know the several Uses and Properties depending thereon, or the
Nature of the thing.



\paragraph{66.} Hence it is evident, that those things which under the Notion of
a Cause cooperating or concurring to the Production of Effects,
are altogether inexplicable, and run us into great Absurdities,
may be very naturally explained, and have a proper and obvious
use assigned them, when they are considered only as Marks or
Signs for our Information.  And it is the searching after, and
endeavouring to understand those Signs instituted by the Author
of Nature, that ought to be the Employment of the Natural
Philosopher, and not the pretending to explain things by
Corporeal Causes; which Doctrine seems to have too much estranged
the Minds of Men from that active Principle, that supreme and
wise Spirit, \emph{in whom we live, move, and have our being}.



\paragraph{67.} In the twelfth place, it may perhaps be objected, that though it
be clear from what has been said, that there can be no such thing
as an inert, senseless, extended, solid, figured, moveable
Substance, existing without the Mind, such as Philosophers
describe Matter: Yet if any Man shall leave out of his Idea of
\emph{Matter}, the positive Ideas of Extension, Figure,
Solidity and Motion, and say that he means only by that Word, an
inert senseless Substance, that exists without the Mind, or
unperceived, which is the Occasion of our Ideas, or at the
presence whereof God is pleased to excite Ideas in us: It doth
not appear, but that Matter taken in this sense may possibly
exist.  In Answer to which I say, First, that it seems no less
absurd to suppose a Substance without Accidents, than it is to
suppose Accidents without a Substance.  But Secondly, though we
should grant this unknown Substance may possibly exist, yet where
can it be supposed to be?  That it exists not in the Mind is
agreed, and that it exists not in Place is no less certain; since
all Extension exists only in the Mind, as hath been already
proved.  It remains therefore that it exists no where at all.



\paragraph{68.} Let us examine a little the Description that is here given us of
\emph{Matter}.  It neither acts, nor perceives, nor is
perceived: For this is all that is meant by saying it is an
inert, senseless, unknown substance; which is a Definition
intirely made up of Negatives, excepting only the relative Notion
of its standing under or supporting: But then it must be
observed, that it \emph{supports} nothing at all; and how nearly
this comes to the Description of a \emph{non-entity}, I desire may
be considered.  But, say you, it is the \emph{unknown
Occasion}, at the presence of which, Ideas are excited in us
by the Will of God.  Now I would fain know how any thing can be
present to us, which is neither perceivable by Sense nor
Reflexion, nor capable of producing any Idea in our Minds, nor is
at all extended, nor hath any Form, nor exists in any Place.  The
Words \emph{to be present}, when thus applied, must needs be
taken in some abstract and strange Meaning, and which I am not
able to comprehend.



\paragraph{69.} Again, let us examine what is meant by \emph{Occasion}: So far
as I can gather from the common use of Language, that Word
signifies, either the Agent which produces any Effect, or else
something that is observed to accompany, or go before it, in the
ordinary Course of things.  But when it is applied to Matter as
above described, it can be taken in neither of those senses.  For
Matter is said to be passive and inert, and so cannot be an Agent
or efficient Cause.  It is also unperceivable, as being devoid of
all sensible Qualities, and so cannot be the Occasion of our
Perceptions in the latter Sense: As when the burning my Finger is
said to be the Occasion of the Pain that attends it.  What
therefore can be meant by calling Matter an \emph{Occasion}?
This Term is either used in no sense at all, or else in some
sense very distant from its received Signification.



\paragraph{70.} You will perhaps say that Matter, though it be not perceived by
us, is nevertheless perceived by {\sc God}, to whom it is the Occasion
of exciting Ideas in our Minds.  For, say you, since we observe
our Sensations to be imprinted in an orderly and constant manner,
it is but reasonable to suppose there are certain constant and
regular Occasions of their being produced.  That is to say, that
there are certain permanent and distinct Parcels of Matter,
corresponding to our Ideas, which, though they do not excite them
in our Minds, or any ways immediately affect us, as being
altogether passive and unperceivable to Us, they are nevertheless
to {\sc God}, by whom they are perceived, as it were so many Occasions
to remind him when and what Ideas to imprint on our Minds: that
so things may go on in a constant uniform manner.



\paragraph{71.} In answer to this I observe, that as the Notion of Matter is here
stated, the Question is no longer concerning the Existence of a
thing distinct from \emph{Spirit} and \emph{Idea}, from
perceiving and being perceived: But whether there are not certain
Ideas, of I know not what sort, in the Mind of {\sc God}, which are so
many Marks or Notes that direct him how to produce Sensations in
our Minds, in a constant and regular Method: Much after the same
manner as a Musician is directed by the Notes of Musick to
produce that harmonious Train and Composition of Sound, which is
called a \emph{Tune}; though they who hear the Musick do not
perceive the Notes, and may be intirely ignorant of them.  But
this Notion of Matter seems too extravagant to deserve a
Confutation.  Besides, it is in effect no Objection against what
we have advanced, to wit, that there is no senseless, unperceived
\emph{Substance}.



\paragraph{72.} If we follow the Light of Reason, we shall, from the constant
uniform Method of our Sensations, collect the Goodness and Wisdom
of the \emph{Spirit} who excites them in our Minds.  But this
is all that I can see reasonably concluded from thence.  To me, I
say, it is evident that the Being of a \emph{Spirit infinitely
Wise, Good, and Powerful} is abundantly sufficient to explain
all the Appearances of Nature.  But as for \emph{inert senseless
Matter}, nothing that I perceive has any the least Connexion
with it, or leads to the Thoughts of it.  And I would fain see
any one explain any the meanest \emph{Ph{\ae}nomenon} in
Nature by it, or shew any manner of Reason, though in the lowest
Rank of Probability, that he can have for its Existence; or even
make any tolerable Sense or Meaning of that Supposition.  For as
to its being an Occasion, we have, I think, evidently shewn that
with regard to us it is no Occasion: It remains therefore that it
must be, if at all, the Occasion to {\sc God} of exciting Ideas in us;
and what this amounts to, we have just now seen.



\paragraph{73.} It is worth while to reflect a little on the Motives which
induced Men to suppose the Existence of material Substance; that
so having observed the gradual ceasing, and Expiration of those
Motives or Reasons, we may proportionably withdraw the Assent
that was grounded on them.  First therefore, it was thought that
Colour, Figure, Motion, and the rest of the sensible Qualities or
Accidents, did really exist without the Mind; and for this
reason, it seemed needful to suppose some unthinking
\emph{Substratum} or \emph{Substance} wherein they did exist,
since they could not be conceived to exist by themselves.
Afterwards, in process of time, Men being convinced that Colours,
Sounds, and the rest of the sensible secondary Qualities had no
Existence without the Mind, they stripped this
\emph{Substratum} or material Substance of those Qualities,
leaving only the primary ones, Figure, Motion, and such like,
which they still conceived to exist without the Mind, and
consequently to stand in need of a material Support.  But it
having been shewn, that none, even of these, can possibly exist
otherwise than in a Spirit or Mind which perceives them, it
follows that we have no longer any reason to suppose the being of
\emph{Matter}.  Nay, that it is utterly impossible there should
be any such thing, so long as that Word is taken to denote an
\emph{unthinking Substratum} of Qualities or Accidents, wherein
they exist without the Mind.



\paragraph{74.} But though it be allowed by the \emph{Materialists} themselves,
that Matter was thought of only for the sake of supporting
Accidents; and the reason intirely ceasing, one might expect the
Mind should naturally, and without any reluctance at all, quit
the belief of what was solely grounded thereon.  Yet the
Prejudice is riveted so deeply in our Thoughts, that we can
scarce tell how to part with it, and are therefore inclined,
since the \emph{Thing} it self is indefensible, at least to
retain the \emph{Name}; which we apply to I know not what
abstracted and indefinite Notions of \emph{Being}, or
\emph{Occasion}, though without any shew of Reason, at least so
far as I can see.  For what is there on our part, or what do we
perceive amongst all the Ideas, Sensations, Notions, which are
imprinted on our Minds, either by Sense or Reflexion, from whence
may be inferred the Existence of an inert, thoughtless,
unperceived Occasion? and on the other hand, on the part of an
\emph{all-sufficient Spirit}, what can there be that should
make us believe, or even suspect, he is \emph{directed} by an
inert Occasion to excite Ideas in our Minds?



\paragraph{75.} It is a very extraordinary Instance of the force of Prejudice,
and much to be lamented, that the Mind of Man retains so great a
Fondness against all the evidence of Reason, for a stupid
thoughtless \emph{Somewhat}, by the interposition whereof it
would, as it were, skreen it self from the Providence of God, and
remove him farther off from the Affairs of the World.  But though
we do the utmost we can, to secure the belief of \emph{Matter},
though when Reason forsakes us, we endeavour to support our
Opinion on the bare possibility of the Thing, and though we
indulge our selves in the full Scope of an Imagination not
regulated by Reason, to make out that poor \emph{Possibility},
yet the upshot of all is, that there are certain \emph{unknown
Ideas} in the Mind of God; for this, if any thing, is all
that I conceive to be meant by \emph{Occasion} with regard to
God.  And this, at the Bottom, is no longer contending for the
\emph{Thing}, but for the \emph{Name}.



\paragraph{76.} Whether therefore there are such Ideas in the Mind of {\sc God}, and
whether they may be called by the name \emph{Matter}, I shall
not dispute.  But if you stick to the Notion of an unthinking
Substance, or Support of Extension, Motion, and other sensible
Qualities, then to me it is most evidently impossible there
should be any such thing.  Since it is a plain Repugnancy, that
those Qualities should exist in or be supported by an
unperceiving Substance.



\paragraph{77.} But say you, though it be granted that there is no thoughtless
support of Extension, and the other Qualities or Accidents which
we perceive; yet there may, perhaps, be some inert unperceiving
Substance, or \emph{Substratum} of some other Qualities, as
incomprehensible to us as Colours are to a Man born blind,
because we have not a Sense adapted to them.  But if we had a new
Sense, we should possibly no more doubt of their Existence, than
a Blind-man made to see does of the Existence of Light and
Colours.  I answer, First, if what you mean by the word
\emph{Matter} be only the unknown Support of unknown Qualities,
it is no matter whether there is such a thing or no, since it no
way concerns us: And I do not see the Advantage there is in
disputing about what we know not \emph{what}, and we know not
\emph{why}.



\paragraph{78.} But secondly, if we had a new Sense, it could only furnish us
with new Ideas or Sensations: And then we should have the same
reason against their existing in an unperceiving Substance, that
has been already offered with relation to Figure, Motion, Colour,
and the like.  Qualities, as hath been shewn, are nothing else
but \emph{Sensations} or \emph{Ideas}, which exist only in a
\emph{Mind} perceiving them; and this is true not only of the
Ideas we are acquainted with at present, but likewise of all
possible Ideas whatsoever.



\paragraph{79.} But you will insist, what if I have no reason to believe the
Existence of Matter, what if I cannot assign any use to it, or
explain any thing by it, or even conceive what is meant by that
Word?  Yet still it is no Contradiction to say that Matter
exists, and that this Matter is \emph{in general} a
\emph{Substance}, or \emph{Occasion of Ideas}; though,
indeed, to go about to unfold the meaning, or adhere to any
particular Explication of those Words, may be attended with great
Difficulties.  I answer, when Words are used without a Meaning,
you may put them together as you please, without danger of
running into a Contradiction.  You may say, for Example, that
\emph{twice Two} is equal to \emph{Seven}, so long as you
declare you do not take the Words of that Proposition in their
usual Acceptation, but for Marks of you know not what.  And by
the same reason you may say, there is an inert thoughtless
Substance without Accidents, which is the occasion of our Ideas.
And we shall understand just as much by one Proposition, as the
other.



\paragraph{80.} In the last place, you will say, What if we give up the Cause of
material Substance, and assert, that Matter is an unknown
\emph{Somewhat}, neither Substance nor Accident, Spirit nor
Idea, inert, thoughtless, indivisible, immoveable, unextended,
existing in no Place?  For, say you, Whatever may be urged
against \emph{Substance} or \emph{Occasion}, or any other
positive or relative Notion of Matter, hath no place at all, so
long as this \emph{negative} Definition of Matter is adhered
to.  I answer, you may, if so it shall seem good, use the word
\emph{Matter} in the same Sense, that other Men use
\emph{nothing}, and so make those Terms convertible in your
Style.  For after all, this is what appears to me to be the
Result of that Definition, the Parts whereof when I consider with
Attention, either collectively, or separate from each other, I do
not find that there is any kind of Effect or Impression made on
my Mind, different from what is excited by the Term
\emph{Nothing}.



\paragraph{81.} You will reply perhaps, that in the foresaid Definition is
included, what doth sufficiently distinguish it from nothing, the
positive, abstract Idea of \emph{Quiddity}, \emph{Entity}, or
\emph{Existence}.  I own indeed, that those who pretend to the
Faculty of framing abstract general Ideas, do talk as if they had
such an Idea, which is, say they, the most abstract and general
Notion of all, that is to me the most incomprehensible of all
others.  That there are a great variety of Spirits of different
Orders and Capacities, whose Faculties, both in Number and
Extent, are far exceeding those the Author of my Being has
bestowed on me, I see no reason to deny.  And for me to pretend
to determine by my own few, stinted, narrow Inlets of Perception,
what Ideas the inexhaustible Power of the {\sc Supreme Spirit} may
imprint upon them, were certainly the utmost Folly and
Presumption.  Since there may be, for ought that I know,
innumerable sorts of Ideas or Sensations, as different from one
another, and from all that I have perceived, as Colours are from
Sounds.  But how ready soever I may be, to acknowledge the
Scantiness of my Comprehension, with regard to the endless
variety of Spirits and Ideas, that might possibly exist, yet for
any one to pretend to a Notion of Entity or Existence,
\emph{abstracted} from \emph{Spirit} and \emph{Idea}, from
perceived and being perceived, is, I suspect, a downright
repugnancy and trifling with Words.  It remains that we consider
the Objections, which may possibly be made on the part of
Religion.



\paragraph{82.} Some there are who think, that though the Arguments for the real
Existence of Bodies, which are drawn from Reason, be allowed not
to amount to Demonstration, yet the Holy Scriptures are so clear
in the Point, as will sufficiently convince every good Christian,
that Bodies do really exist, and are something more than mere
Ideas; there being in Holy Writ innumerable Facts related, which
evidently suppose the reality of Timber, and Stone, Mountains,
and Rivers, and Cities, and humane Bodies.  To which I answer,
that no sort of Writings whatever, sacred or profane, which use
those and the like Words in the vulgar Acceptation, or so as to
have a meaning in them, are in danger of having their Truth
called in question by our Doctrine.  That all those Things do
really exist, that there are Bodies, even corporeal Substances,
when taken in the vulgar Sense, has been shewn to be agreeable to
our Principles: And the difference betwixt \emph{Things} and
\emph{Ideas}, \emph{Realities} and \emph{Chimeras}, has
been distinctly explained.\authornote{Sect.\ 29, 30, 33, 36, \&c.}
And I do not think, that either what Philosophers call
\emph{Matter}, or the Existence of Objects without the Mind, is
any where mentioned in Scripture.



\paragraph{83.} Again, whether there be, or be not external Things, it is agreed
on all hands, that the proper Use of Words, is the marking our
Conceptions, or Things only as they are known and perceived by
us; whence it plainly follows, that in the Tenets we have laid
down, there is nothing inconsistent with the right Use and
Significancy of \emph{Language}, and that Discourse of what
kind soever, so far as it is intelligible, remains undisturbed.
But all this seems so manifest, from what hath been set forth in
the Premises, that it is needless to insist any farther on it.



\paragraph{84.} But it will be urged, that Miracles do, at least, lose much of
their Stress and Import by our Principles.  What must we think of
\emph{Moses}'s Rod, was it not \emph{really} turned into a
Serpent, or was there only a Change of \emph{Ideas} in the
Minds of the Spectators?  And can it be supposed, that our
Saviour did no more at the Marriage-Feast in \emph{Cana}, than
impose on the Sight, and Smell, and Taste of the Guests, so as to
create in them the Appearance or Idea only of Wine? The same may
be said of all other Miracles: Which, in consequence of the
foregoing Principles, must be looked upon only as so many Cheats,
or Illusions of Fancy.  To this I reply, that the Rod was changed
into a real Serpent, and the Water into real Wine.  That this
doth not, in the least, contradict what I have elsewhere said,
will be evident from \emph{Sect.}\ 34, and 35.
But this Business of \emph{Real} and \emph{Imaginary} hath
been already so plainly and fully explained, and so often
referred to, and the Difficulties about it are so easily answered
from what hath gone before, that it were an Affront to the
Reader's Understanding, to resume the Explication of it in this
place.  I shall only observe, that if at Table all who were
present should see, and smell, and taste, and drink Wine, and
find the effects of it, with me there could be no doubt of its
Reality.  So that, at Bottom, the Scruple concerning real
Miracles hath no place at all on ours, but only on the received
Principles, and consequently maketh rather \emph{for}, than
\emph{against} what hath been said.



\paragraph{85.} Having done with the Objections, which I endeavoured to propose
in the clearest Light, and gave them all the Force and Weight I
could, we proceed in the next place to take a view of our Tenets
in their Consequences.  Some of these appear at first Sight, as
that several difficult and obscure Questions, on which abundance
of Speculation hath been thrown away, are intirely banished from
Philosophy.  Whether corporeal Substance can think?  Whether
Matter be infinitely divisible?  And how it operates on Spirit?
these and like Inquiries have given infinite Amusement to
Philosophers in all Ages.  But depending on the Existence of
\emph{Matter}, they have no longer any place on our Principles.
Many other Advantages there are, as well with regard to
\emph{Religion} as the \emph{Sciences}, which it is easy for
any one to deduce from what hath been premised.  But this will
appear more plainly in the Sequel.



\paragraph{86.} From the Principles we have laid down, it follows, humane
Knowledge may naturally be reduced to two Heads, that of
\emph{Ideas}, and that of \emph{Spirits}.  Of each of these I
shall treat in order.  And first as to Ideas or unthinking
Things, our Knowledge of these hath been very much obscured and
confounded, and we have been led into very dangerous Errors, by
supposing a twofold Existence of the Objects of Sense, the one
\emph{intelligible}, or in the Mind, the other \emph{real}
and without the Mind: Whereby unthinking Things are thought to
have a natural Subsistence of their own, distinct from being
perceived by Spirits.  This which, if I mistake not, hath been
shewn to be a most groundless and absurd Notion, is the very Root
of \emph{Scepticism}; for so long as Men thought that real
Things subsisted without the Mind, and that their Knowledge was
only so far forth \emph{real} as it was conformable to \emph{real
Things}, it follows, they could not be certain they had any
real Knowledge at all.  For how can it be known, that the Things
which are perceived, are conformable to those which are not
perceived, or exist without the Mind?



\paragraph{87.} Colour, Figure, Motion, Extension and the like, considered only
as so many \emph{Sensations} in the Mind, are perfectly known,
there being nothing in them which is not perceived.  But if they
are looked on as Notes or Images, referred to \emph{Things} or
\emph{Archetypes} existing without the Mind, then are we
involved all in \emph{Scepticism}.  We see only the
Appearances, and not the real Qualities of Things.  What may be
the Extension, Figure, or Motion of any thing really and
absolutely, or in it self, it is impossible for us to know, but
only the proportion or the relation they bear to our Senses.
Things remaining the same, our Ideas vary, and which of them, or
even whether any of them at all represent the true Quality really
existing in the Thing, it is out of our reach to determine.  So
that, for ought we know, all we see, hear, and feel, may be only
Phantom and vain Chimera, and not at all agree with the real
Things, existing in \emph{Rerum Natura}.  All this Scepticism
follows, from our supposing a difference between \emph{Things}
and \emph{Ideas}, and that the former have a Subsistence
without the Mind, or unperceived.  It were easy to dilate on this
Subject, and shew how the Arguments urged by \emph{Sceptics} in
all Ages, depend on the Supposition of external Objects.



\paragraph{88.} So long as we attribute a real Existence to unthinking Things,
distinct from their being perceived, it is not only impossible
for us to know with evidence the Nature of any real unthinking
Being, but even that it exists.  Hence it is, that we see
Philosophers distrust their Senses, and doubt of the Existence of
Heaven and Earth, of every thing they see or feel, even of their
own Bodies.  And after all their labour and struggle of Thought,
they are forced to own, we cannot attain to any self-evident or
demonstrative Knowledge of the Existence of sensible Things.  But
all this Doubtfulness, which so bewilders and confounds the Mind,
and makes \emph{Philosophy} ridiculous in the Eyes of the
World, vanishes, if we annex a meaning to our Words, and do not
amuse our selves with the Terms \emph{Absolute},
\emph{External}, \emph{Exist}, and such like, signifying we
know not what.  I can as well doubt of my own Being, as of the
Being of those Things which I actually perceive by Sense: It
being a manifest Contradiction, that any sensible Object should
be immediately perceived by Sight or Touch, and at the same time
have no Existence in Nature, since the very Existence of
an unthinking Being consists in \emph{being perceived}.



\paragraph{89.} Nothing seems of more Importance, towards erecting a firm Systeme
of sound and real Knowledge, which may be proof against the
Assaults of \emph{Scepticism}, than to lay the beginning in a
distinct Explication of what is meant by \emph{Thing},
\emph{Reality}, \emph{Existence}: For in vain shall we
dispute concerning the real Existence of Things, or pretend to
any Knowledge thereof, so long as we have not fixed the meaning
of those Words.  \emph{Thing} or \emph{Being} is the most
general Name of all, it comprehends under it two Kinds intirely
distinct and heterogeneous, and which have nothing common but the
Name, to wit, \emph{Spirits} and \emph{Ideas}.  The former
are \emph{active, indivisible Substances}: The latter are
\emph{inert, fleeting, dependent Beings}, which subsist not by
themselves, but are supported by, or exist in Minds or spiritual
Substances.  We comprehend our own Existence by inward Feeling or
Reflexion, and that of other Spirits by Reason.  We may be said
to have some Knowledge or Notion of our own Minds, of Spirits and
active Beings, whereof in a strict Sense we have not Ideas.  In
like manner we know and have a Notion of relations between Things
or Ideas, which relations are distinct from the Ideas or Things
related, inasmuch as the latter may be perceived by us without
our perceiving the former.  To me it seems that Ideas, Spirits
and Relations are all in their respective kinds, the Object of
humane Knowledge and Subject of Discourse: and that the Term
\emph{Idea} would be improperly extended to signify every thing
we know or have any Notion of.



\paragraph{90.} Ideas imprinted on the Senses are real Things, or do really
exist; this we do not deny, but we deny they can subsist without
the Minds which perceive them, or that they are Resemblances of
any Archetypes existing without the Mind: Since the very Being of
a Sensation or Idea consists in being perceived, and an Idea can
be like nothing but an Idea.  Again, the Things perceived by
Sense may be termed \emph{external}, with regard to their
Origin, in that they are not generated from within, by the Mind
it self, but imprinted by a Spirit distinct from that which
perceives them.  Sensible Objects may likewise be said to be
without the Mind, in another sense, namely when they exist in
some other Mind.  Thus when I shut my Eyes, the Things I saw may
still exist, but it must be in another Mind.



\paragraph{91.} It were a mistake to think, that what is here said derogates in
the least from the Reality of Things.  It is acknowledged on the
received Principles, that Extension, Motion, and in a word all
sensible Qualities, have need of a Support, as not being able to
subsist by themselves.  But the Objects perceived by Sense, are
allowed to be nothing but Combinations of those Qualities, and
consequently cannot subsist by themselves.  Thus far it is agreed
on all hands.  So that in denying the Things perceived by Sense,
an Existence independent of a Substance, or Support wherein they
may exist, we detract nothing from the received Opinion of their
\emph{Reality}, and are guilty of no Innovation in that
respect.  All the difference is, that according to us the
unthinking Beings perceived by Sense, have no Existence distinct
from Being perceived, and cannot therefore exist in any other
Substance, than those unextended, indivisible Substances, or
\emph{Spirits}, which act, and think, and perceive them: Whereas
Philosophers vulgarly hold, that the sensible Qualities exist in
an inert, extended, unperceiving Substance, which they call
\emph{Matter}, to which they attribute a natural Subsistence,
exterior to all thinking Beings, or distinct from Being perceived
by any Mind whatsoever, even the eternal Mind of the {\sc Creator},
wherein they suppose only Ideas of the corporeal Substances
created by him: If indeed they allow them to be at all created.



\paragraph{92.} For as we have shewn the Doctrine of Matter or corporeal
Substance, to have been the main Pillar and Support of
\emph{Scepticism}, so likewise upon the same Foundation have
been raised all the impious Schemes of \emph{Atheism} and
Irreligion.  Nay so great a difficulty hath it been thought, to
conceive Matter produced out of nothing, that the most celebrated
among the ancient Philosophers, even of these who maintained the
Being of a {\sc God}, have thought Matter to be uncreated and coeternal
with him.  How great a Friend material Substance hath been to
\emph{Atheists} in all Ages, were needless to relate.  All
their monstrous Systems have so visible and necessary a
dependence on it, that when this Corner-stone is once removed,
the whole Fabrick cannot choose but fall to the Ground; insomuch
that it is no longer worth while, to bestow a particular
Consideration on the Absurdities of every wretched Sect of
\emph{Atheists}.



\paragraph{93.} That impious and profane Persons should readily fall in with
those Systems which favour their Inclinations, by deriding
immaterial Substance, and supposing the Soul to be divisible and
subject to Corruption as the Body; which exclude all Freedom,
Intelligence, and Design from the Formation of Things, and
instead thereof make a self-existent, stupid, unthinking
Substance the Root and Origin of all Beings.  That they should
hearken to those who deny a Providence, or Inspection of a
superior Mind over the Affairs of the World, attributing the
whole Series of Events either to blind Chance or fatal Necessity,
arising from the Impulse of one Body on another.  All this is
very natural.  And on the other hand, when Men of better
Principles observe the Enemies of Religion lay so great a Stress
on \emph{unthinking Matter}, and all of them use so much
Industry and Artifice to reduce every thing to it; methinks they
should rejoice to see them deprived of their grand Support, and
driven from that only Fortress, without which your
\emph{Epicureans}, \emph{Hobbists}, and the like, have not
even the Shadow of a Pretence, but become the most cheap and easy
Triumph in the World.



\paragraph{94.} The Existence of Matter, or Bodies unperceived, has not only been
the main Support of \emph{Atheists} and \emph{Fatalists}, but
on the same Principle doth \emph{Idolatry} likewise in all its
various Forms depend.  Did Men but consider that the Sun, Moon,
and Stars, and every other Object of the Senses, are only so many
Sensations in their Minds, which have no other Existence but
barely being perceived, doubtless they would never fall down, and
worship their own \emph{Ideas}; but rather address their Homage
to that {\sc Eternal Invisible Mind} which produces and sustains
all Things.



\paragraph{95.} The same absurd Principle, by mingling it self with the Articles
of our Faith, hath occasioned no small Difficulties to
Christians.  For Example, about the \emph{Resurrection}, how
many Scruples and Objections have been raised by
\emph{Socinians} and others? But do not the most plausible of
them depend on the supposition, that a Body is denominated the
\emph{same}, with regard not to the Form or that which is
perceived by Sense, but the material Substance which remains the
same under several Forms?  Take away this \emph{material
Substance}, about the Identity whereof all the Dispute is,
and mean by \emph{Body} what every plain ordinary Person means
by that Word, to wit, that which is immediately seen and felt,
which is only a Combination of sensible Qualities, or Ideas: And
then their most unanswerable Objections come to nothing.



\paragraph{96.} Matter being once expelled out of Nature, drags with it so many
sceptical and impious Notions, such an incredible number of
Disputes and puzling Questions, which have been Thorns in the
Sides of Divines, as well as Philosophers, and made so much
fruitless Work for Mankind; that if the Arguments we have
produced against it, are not found equal to Demonstration (as to
me they evidently seem) yet I am sure all Friends to Knowledge,
Peace, and Religion, have reason to wish they were.



\paragraph{97.} Beside the external Existence of the Objects of Perception,
another great Source of Errors and Difficulties, with regard to
Ideal Knowledge, is the Doctrine of \emph{abstract Ideas}, such
as it hath been set forth in the Introduction.  The plainest
Things in the World, those we are most intimately acquainted
with, and perfectly know, when they are considered in an abstract
way, appear strangely difficult and incomprehensible.  Time,
Place, and Motion, taken in particular or concrete, are what
every Body knows; but having passed through the Hands of a
Metaphysician, they become too abstract and fine, to be
apprehended by Men of ordinary Sense.  Bid your Servant meet you
at such a \emph{Time}, in such a \emph{Place}, and he shall
never stay to deliberate on the meaning of those Words: In
conceiving that particular Time and Place, or the Motion by which
he is to get thither, he finds not the least Difficulty.  But if
\emph{Time} be taken, exclusive of all those particular Actions
and Ideas that diversify the Day, merely for the Continuation of
Existence, or Duration in Abstract, then it will perhaps gravel
even a Philosopher to comprehend it.



\paragraph{98.} Whenever I attempt to frame a simple Idea of \emph{Time},
abstracted from the succession of Ideas in my Mind, which flows
uniformly, and is participated by all Beings, I am lost and
embrangled in inextricable Difficulties.  I have no Notion of it
at all, only I hear others say, it is infinitely divisible, and
speak of it in such a manner as leads me to entertain odd
Thoughts of my Existence: Since that Doctrine lays one under an
absolute necessity of thinking, either that he passes away
innumerable Ages without a Thought, or else that he is
annihilated every moment of his Life: Both which seem equally
absurd.  Time therefore being nothing, abstracted from the
Sucession of Ideas in our Minds, it follows that the Duration of
any finite Spirit must be estimated by the Number of Ideas or
Actions succeeding each other in that same Spirit or Mind.  Hence
it is a plain consequence that the Soul always thinks: And in
truth whoever shall go about to divide in his Thoughts, or
abstract the \emph{Existence} of a Spirit from its
\emph{Cogitation}, will, I believe, find it no easy Task.



\paragraph{99.} So likewise, when we attempt to abstract Extension and Motion
from all other Qualities, and consider them by themselves, we
presently lose sight of them, and run into great Extravagancies.
All which depend on a two-fold Abstraction: First, it is supposed
that Extension, for Example, may be abstracted from all other
sensible Qualities; and Secondly, that the Entity of Extension
may be abstracted from its being perceived.  But whoever shall
reflect, and take care to understand what he says, will, if I
mistake not, acknowledge that all sensible Qualities are alike
\emph{Sensations}, and alike \emph{real}; that where the
Extension is, there is the Colour too, to wit, in his Mind, and
that their Archetypes can exist only in some other \emph{Mind}:
And that the Objects of Sense are nothing but those Sensations
combined, blended, or (if one may so speak) concreted together:
None of all which can be supposed to exist unperceived.



\paragraph{100.} What it is for a Man to be happy, or an Object good, every one
may think he knows.  But to frame an abstract Idea of
\emph{Happiness}, prescinded from all particular Pleasure, or
of \emph{Goodness}, from every thing that is good, this is what
few can pretend to.  So likewise, a Man may be just and virtuous,
without having precise Ideas of \emph{Justice} and
\emph{Virtue}.  The Opinion that those and the like Words stand
for general Notions abstracted from all particular Persons and
Actions, seems to have rendered Morality difficult, and the
Study thereof of less use to Mankind.  And in effect, the
Doctrine of \emph{Abstraction} has not a little contributed
towards spoiling the most useful Parts of Knowledge.



\paragraph{101.} The two great Provinces of speculative Science, conversant about
Ideas received from Sense and their Relations, are \emph{natural
Philosophy} and \emph{Mathematics}; with regard to each of
these I shall make some Observations.  And First, I shall say
somewhat of natural Philosophy.  On this Subject it is, that the
\emph{Sceptics} triumph: All that stock of Arguments they
produce to depreciate our Faculties, and make Mankind appear
ignorant and low, are drawn principally from this Head, to wit,
that we are under an invincible Blindness as to the \emph{true}
and \emph{real} Nature of Things.  This they exaggerate, and
love to enlarge on.  We are miserably bantered, say they, by our
Senses, and amused only with the outside and shew of Things.  The
real Essence, the internal Qualities, and Constitution of every
the meanest Object, is hid from our view; something there is in
every drop of Water, every grain of Sand, which it is beyond the
Power of humane Understanding to fathom or comprehend.  But it is
evident from what has been shewn, that all this Complaint is
groundless, and that we are influenced by false Principles to
that degree as to mistrust our Senses, and think we know nothing
of those Things which we perfectly comprehend.



\paragraph{102.} One great Inducement to our pronouncing our selves ignorant of
the Nature of Things, is the current Opinion that every thing
includes within it self the Cause of its Properties: Or that
there is in each Object an inward Essence, which is the Source
whence its discernible Qualities flow, and whereon they depend.
Some have pretended to account for Appearances by occult
Qualities, but of late they are mostly resolved into mechanical
Causes, to wit, the Figure, Motion, Weight, and such like
Qualities of insensible Particles: Whereas in truth, there is no
other Agent or efficient Cause than \emph{Spirit}, it being
evident that Motion, as well as all other \emph{Ideas}, is
perfectly inert.  See \emph{Sect.}~25.
Hence, to endeavour to explain the Production of Colours or
Sounds, by Figure, Motion, Magnitude and the like, must needs be
labour in vain.  And accordingly, we see the Attempts of that
kind are not at all satisfactory.  Which may be said, in general,
of those Instances, wherein one Idea or Quality is assigned for
the Cause of another.  I need not say, how many
\emph{Hypotheses} and Speculations are left out, and how much
the Study of Nature is abridged by this Doctrine.



\paragraph{103.} The great mechanical Principle now in Vogue is
\emph{Attraction}.  That a Stone falls to the Earth, or the Sea
swells towards the Moon, may to some appear sufficiently
explained thereby.  But how are we enlightened by being told this
is done by Attraction? Is it that that Word signifies the manner
of the Tendency, and that it is by the mutual drawing of Bodies,
instead of their being impelled or protruded towards each other?
But nothing is determined of the Manner or Action, and it may as
truly (for ought we know) be termed \emph{Impulse} or
\emph{Protrusion} as \emph{Attraction}.  Again, the Parts of
Steel we see cohere firmly together, and this also is accounted
for by Attraction; but in this, as in the other Instances, I do
not perceive that any thing is signified besides the Effect it
self; for as to the manner of the Action whereby it is produced,
or the Cause which produces it, these are not so much as aimed
at.



\paragraph{104.} Indeed, if we take a view of the several
\emph{Ph{\ae}nomena}, and compare them together, we may
observe some likeness and conformity between them.  For Example,
in the falling of a Stone to the Ground, in the rising of the Sea
towards the Moon, in Cohesion and Crystallization, there is
something alike, namely an Union or mutual Approach of Bodies.
So that any one of these or the like \emph{Ph{\ae}nomena},
may not seem strange or surprising to a Man who hath nicely
observed and compared the Effects of Nature.  For that only is
thought so which is uncommon, or a thing by it self, and out of
the ordinary Course of our Observation.  That Bodies should tend
towards the Center of the Earth, is not thought strange, because
it is what we perceive every moment of our Lives.  But that they
should have a like Gravitation towards the Center of the Moon,
may seem odd and unaccountable to most Men, because it is
discerned only in the Tides.  But a Philosopher, whose Thoughts
take in a larger compass of Nature, having observed a certain
similitude of Appearances, as well in the Heavens as the Earth,
that argue innumerable Bodies to have a mutual Tendency towards
each other, which he denotes by the general Name
\emph{Attraction}, whatever can be reduced to that, he thinks
justly accounted for.  Thus he explains the Tides by the
Attraction of the Terraqueous Globe towards the Moon, which to
him doth not appear odd or anomalous, but only a particular
Example of a general Rule or Law of Nature.



\paragraph{105.} If therefore we consider the difference there is betwixt natural
Philosophers and other Men, with regard to their Knowledge of the
\emph{Ph{\ae}nomena}, we shall find it consists, not in an
exacter Knowledge of the efficient Cause that produces them,  for
that can be no other than the \emph{Will of a Spirit},  but
only in a greater Largeness of Comprehension, whereby Analogies,
Harmonies, and Agreements are discovered in the Works of Nature,
and the particular Effects explained, that is, reduced to general
Rules, see \emph{Sect.}~62.\ which Rules
grounded on the Analogy, and Uniformness observed in the
Production of natural Effects, are most agreeable, and sought
after by the Mind; for that they extend our Prospect beyond what
is present, and near to us, and enable us to make very probable
Conjectures, touching Things that may have happened at very great
distances of Time and Place, as well as to predict Things to
come; which sort of endeavour towards Omniscience, is much
affected by the Mind.



\paragraph{106.} But we should proceed warily in such Things: for we are apt to
lay too great a Stress on Analogies, and to the prejudice of
Truth, humour that Eagerness of the Mind, whereby it is carried
to extend its Knowledge into general Theoremes.  For Example,
Gravitation, or mutual Attraction, because it appears in many
Instances, some are straightway for pronouncing
\emph{Universal}; and that to \emph{attract, and be attracted by
every other Body, is an essential Quality inherent in all Bodies
whatsoever}.  Whereas it appears the fixed Stars have no such
Tendency towards each other: and so far is that Gravitation, from
being \emph{essential} to Bodies, that in some Instances a
quite contrary Principle seems to shew it self: As in the
perpendicular Growth of Plants, and the Elasticity of the Air.
There is nothing necessary or essential in the Case, but it
depends intirely on the Will of the \emph{governing Spirit},
who causes certain Bodies to cleave together, or tend towards
each other, according to various Laws, whilst he keeps others at
a fixed Distance; and to some he gives a quite contrary Tendency
to fly asunder, just as he sees convenient.



\paragraph{107.} After what has been premised, I think we may lay down the
following Conclusions.  First, It is plain Philosophers amuse
themselves in vain, when they inquire for any natural efficient
Cause, distinct from a \emph{Mind} or \emph{Spirit}.
Secondly, Considering the whole Creation is the Workmanship of a
\emph{wise and good Agent}, it should seem to become
Philosophers, to employ their Thoughts (contrary to what some
hold) about the final Causes of Things: And I must confess, I see
no reason, why pointing out the various Ends, to which natural
Things are adapted, and for which they were originally with
unspeakable Wisdom contrived, should not be thought one good way
of accounting for them, and altogether worthy a Philosopher.
Thirdly, From what hath been premised no reason can be drawn, why
the History of Nature should not still be studied, and
Observations and Experiments made, which, that they are of use to
Mankind, and enable us to draw any general Conclusions, is not
the Result of any immutable Habitudes, or Relations between
Things themselves, but only of {\sc God}'s Goodness and Kindness to Men
in the Administration of the World.  See
\emph{Sect.}\ 30 and 31.
Fourthly, By a diligent Observation of the
\emph{Ph{\ae}nomena} within our View, we may discover the
general Laws of Nature, and from them deduce the other
\emph{Ph{\ae}nomena}, I do not say \emph{demonstrate}; for
all Deductions of that kind depend on a Supposition that the
Author of Nature always operates uniformly, and in a constant
observance of those Rules we take for Principles: Which we cannot
evidently know.



\paragraph{108.} Those Men who frame general Rules from the
\emph{Ph{\ae}nomena}, and afterwards derive the
\emph{Ph{\ae}nomena} from those Rules, seem to consider Signs
rather than Causes.  A Man may well understand natural Signs
without knowing their Analogy, or being able to say by what Rule
a Thing is so or so.  And as it is very possible to write
improperly, through too strict an Observance of general
Grammar-Rules: So in arguing from general Rules of Nature, it is
not impossible we may extend the Analogy too far, and by that
means run into Mistakes.



\paragraph{109.} As in reading other Books, a wise Man will choose to fix his
Thoughts on the Sense and apply it to use, rather than lay them
out in Grammatical Remarks on the Language; so in perusing the
Volume of Nature, it seems beneath the Dignity of the Mind to
affect an Exactness in reducing each particular
\emph{Ph{\ae}nomenon} to general Rules, or shewing how it
follows from them.  We should propose to ourselves nobler Views,
such as to recreate and exalt the Mind, with a prospect of the
Beauty, Order, Extent, and Variety of natural Things: Hence, by
proper Inferences, to enlarge our Notions of the Grandeur,
Wisdom, and Beneficence of the {\sc Creator}: And lastly, to make the
several Parts of the Creation, so far as in us lies, subservient
to the Ends they were designed for, {\sc God}'s Glory, and the
Sustentation and Comfort of our selves and Fellow-Creatures.



\paragraph{110.} The best Key for the aforesaid Analogy, or natural Science, will
be easily acknowledged to be a certain celebrated Treatise of
\emph{Mechanics}: In the entrance of which justly admired
Treatise, Time, Space, and Motion, are distinguished into
\emph{Absolute} and \emph{Relative}, \emph{True} and
\emph{Apparent}, \emph{Mathematical} and \emph{Vulgar}:
Which Distinction, as it is at large explained by the Author,
doth suppose those Quantities to have an Existence without the
Mind: And that they are ordinarily conceived with relation to
sensible Things, to which nevertheless in their own Nature, they
bear no relation at all.



\paragraph{111.} As for \emph{Time}, as it is there taken in an absolute or
abstracted Sense, for the Duration or Perseverance of the
Existence of Things, I have nothing more to add concerning it,
after what hath been already said on that Subject,
\emph{Sect.}\ 97 and 98.
For the rest, this celebrated Author holds there is an
\emph{absolute Space}, which, being unperceivable to Sense,
remains in it self similar and immoveable: And relative Space to
be the measure thereof, which being moveable, and defined by its
Situation in respect of sensible Bodies, is vulgarly taken for
immoveable Space.  \emph{Place} he defines to be that part of
Space which is occupied by any Body.  And according as the Space
is absolute or relative, so also is the Place.  \emph{Absolute
Motion} is said to be the Translation of a Body from absolute
Place to absolute Place, as relative Motion is from one relative
Place to another.  And because the Parts of absolute Space, do
not fall under our Senses, instead of them we are obliged to use
their sensible Measures: And so define both Place and Motion with
respect to Bodies, which we regard as immoveable.  But it is
said, in philosophical Matters we must abstract from our Senses,
since it may be, that none of those Bodies which seem to be
quiescent, are truly so: And the same thing which is moved
relatively, may be really at rest.  As likewise one and the same
Body may be in relative Rest and Motion, or even moved with
contrary relative Motions at the same time, according as its
Place is variously defined.  All which Ambiguity is to be found
in the apparent Motions, but not at all in the true or absolute,
which should therefore be alone regarded in Philosophy.  And the
true, we are told, are distinguished from apparent or relative
Motions by the following Properties.  First, In true or absolute
Motion, all Parts which preserve the same Position with respect
to the whole, partake of the Motions of the whole.  Secondly, The
Place being moved, that which is placed therein is also moved: So
that a Body moving in a Place which is in Motion, doth
participate the Motion of its Place.  Thirdly, True Motion is
never generated or changed, otherwise than by Force impressed on
the Body it self.  Fourthly, True Motion is always changed by
Force impressed on the Body moved.  Fifthly, In circular Motion
barely relative, there is no centrifugal Force, which
nevertheless in that which is true or absolute, is proportional
to the Quantity of Motion.



\paragraph{112.} But notwithstanding what hath been said, it doth not appear to
me, that there can be any Motion other than \emph{relative}: So
that to conceive Motion, there must be at least conceived two
Bodies, whereof the Distance or Position in regard to each other
is varied.  Hence if there was one only Body in being, it could
not possibly be moved.  This seems evident, in that the Idea I
have of Motion doth necessarily include Relation.



\paragraph{113.} But though in every Motion it be necessary to conceive more
Bodies than one, yet it may be that one only is moved, namely
that on which the Force causing the change of distance is
impressed, or in other Words, that to which the Action is
applied.  For however some may define Relative Motion, so as to
term that Body \emph{moved}, which changes its Distance from
some other Body, whether the Force or Action causing that Change
were applied to it, or no: Yet as Relative Motion is that which
is perceived by Sense, and regarded in the ordinary Affairs of
Life, it should seem that every Man of common Sense knows what it
is, as well as the best Philosopher: Now I ask any one, whether
in his Sense of Motion as he walks along the Streets, the Stones
he passes over may be said to \emph{move}, because they change
Distance with his Feet? To me it seems, that though Motion
includes a Relation of one thing to another, yet it is not
necessary that each Term of the Relation be denominated from it.
As a Man may think of somewhat which doth not think, so a Body
may be moved to or from another Body, which is not therefore it
self in Motion.



\paragraph{114.} As the Place happens to be variously defined, the Motion which is
related to it varies.  A Man in a Ship may be said to be
quiescent, with relation to the sides of the Vessel, and yet move
with relation to the Land.  Or he may move Eastward in respect of
the one, and Westward in respect of the other.  In the common
Affairs of Life, Men never go beyond the Earth to define the
Place of any Body: And what is quiescent in respect of that, is
accounted \emph{absolutely} to be so.  But Philosophers who
have a greater Extent of Thought, and juster Notions of the
System of Things, discover even the Earth it self to be moved.
In order therefore to fix their Notions, they seem to conceive
the Corporeal World as finite, and the utmost unmoved Walls or
Shell thereof to be the Place, whereby they estimate true
Motions.  If we sound our own Conceptions, I believe we may find
all the absolute Motion we can frame an Idea of, to be at bottom
no other than relative Motion thus defined.  For as hath been
already observed, absolute Motion exclusive of all external
Relation is incomprehensible: And to this kind of Relative
Motion, all the above-mentioned Properties, Causes, and Effects
ascribed to absolute Motion, will, if I mistake not, be found to
agree.  As to what is said of the centrifugal Force, that it doth
not at all belong to circular Relative Motion: I do not see how
this follows from the Experiment which is brought to prove it.
See \emph{Philosophiae Naturalis Principia Mathematica, in
Schol.\ Def.~VIII}.  For the Water in the Vessel, at that time
wherein it is said to have the greatest relative circular Motion,
hath, I think, no Motion at all: As is plain from the foregoing
Section.



\paragraph{115.} For to denominate a Body \emph{moved}, it is requisite, first,
that it change its Distance or Situation with regard to some
other Body: And secondly, that the Force or Action occasioning
that Change be applied to it.  If either of these be wanting, I
do not think that agreeably to the Sense of Mankind, or the
Propriety of Language, a Body can be said to be in Motion.  I
grant indeed, that it is possible for us to think a Body, which
we see change its Distance from some other, to be moved, though
it have no force applied to it, (in which Sense there may be
apparent Motion,) but then it is, because the Force causing the
Change of Distance, is imagined by us to be applied or impressed
on that Body thought to move.  Which indeed shews we are capable
of mistaking a thing to be in Motion which is not, and that is
all.



\paragraph{116.} From what hath been said, it follows that the Philosophic
Consideration of Motion doth not imply the being of an
\emph{absolute Space}, distinct from that which is perceived by
Sense, and related to Bodies: Which that it cannot exist without
the Mind, is clear upon the same Principles, that demonstrate the
like of all other Objects of Sense.  And perhaps, if we inquire
narrowly, we shall find we cannot even frame an Idea of \emph{pure
Space}, exclusive of all Body.  This I must confess seems
impossible, as being a most abstract Idea.  When I excite a
Motion in some part of my Body, if it be free or without
Resistance, I say there is \emph{Space}: But if I find a
Resistance, then I say there is \emph{Body}: and in proportion
as the Resistance to Motion is lesser or greater, I say the
\emph{Space} is more or less \emph{pure}.  So that when I
speak of pure or empty Space, it is not to be supposed, that the
Word \emph{Space} stands for an Idea distinct from, or
conceivable without Body and Motion.  Though indeed we are apt to
think every Noun Substantive stands for a distinct Idea, that may
be separated from all others: Which hath occasioned infinite
Mistakes.  When therefore supposing all the World to be
annihilated besides my own Body, I say there still remains
\emph{pure Space}: Thereby nothing else is meant, but only that
I conceive it possible, for the Limbs of my Body to be moved on
all sides without the least Resistance: But if that too were
annihilated, then there could be no Motion, and consequently no
Space.  Some perhaps may think the Sense of Seeing doth furnish
them with the Idea of pure Space; but it is plain from what we
have elsewhere shewn, that the Ideas of Space and Distance are
not obtained by that Sense.  See the \emph{Essay concerning
Vision}.



\paragraph{117.} What is here laid down, seems to put an end to all those Disputes
and Difficulties, which have sprung up amongst the Learned
concerning the nature of \emph{pure Space}.  But the chief
Advantage arising from it, is, that we are freed from that
dangerous \emph{Dilemma}, to which several who have employed
their Thoughts on this Subject, imagine themselves reduced, to
wit, of thinking either that Real Space is {\sc God}, or else that
there is something beside {\sc God} which is Eternal, Uncreated,
Infinite, Indivisible, Immutable.  Both which may justly be
thought pernicious and absurd Notions.  It is certain that not a
few Divines, as well as Philosophers of great note, have, from
the Difficulty they found in conceiving either Limits or
Annihilation of Space, concluded it must be \emph{Divine}.  And
some of late have set themselves particularly to shew, that the
incommunicable Attributes of {\sc God} agree to it.  Which Doctrine,
how unworthy soever it may seem of the Divine Nature, yet I do
not see how we can get clear of it, so long as we adhere to the
received Opinions.



\paragraph{118.} Hitherto of Natural Philosophy: We come now to make some Inquiry
concerning that other great Branch of speculative Knowledge, to
wit, \emph{Mathematics}.  These, how celebrated soever they may
be, for their Clearness and Certainty of Demonstration, which is
hardly any where else to be found, cannot nevertheless be
supposed altogether free from Mistakes; if in their Principles
there lurks some secret Error, which is common to the Professors
of those Sciences with the rest of Mankind.  Mathematicians,
though they deduce their Theoremes from a great height of
Evidence, yet their first Principles are limited by the
consideration of Quantity: And they do not ascend into any
Inquiry concerning those transcendental Maxims, which influence
all the particular Sciences, each Part whereof, Mathematics not
excepted, doth consequently participate of the Errors involved in
them.  That the Principles laid down by Mathematicians are true,
and their way of Deduction from those Principles clear and
incontestible, we do not deny.  But we hold, there may be certain
erroneous Maxims of greater Extent than the Object of
Mathematics, and for that reason not expresly mentioned, though
tacitly supposed throughout the whole progress of that Science;
and that the ill Effects of those secret unexamined Errors are
diffused through all the Branches thereof.  To be plain, we
suspect the Mathematicians are, as well as other Men, concerned
in the Errors arising from the Doctrine of abstract general
Ideas, and the Existence of Objects without the Mind.



\paragraph{119.} \emph{Arithmetic} hath been thought to have for its Object
abstract Ideas of \emph{Number}.  Of which to understand the
Properties and mutual Habitudes is supposed no mean part of
speculative Knowledge.  The Opinion of the pure and intellectual
Nature of Numbers in Abstract, hath made them in esteem with
those Philosophers, who seem to have affected an uncommon
Fineness and Elevation of Thought.  It hath set a Price on the
most trifling numerical Speculations which in Practice are of no
use, but serve only for Amusement: And hath therefore so far
infected the Minds of some, that they have dreamed of mighty
\emph{Mysteries} involved in Numbers, and attempted the
Explication of natural Things by them.  But if we inquire into
our own Thoughts, and consider what hath been premised, we may
perhaps entertain a low Opinion of those high Flights and
Abstractions, and look on all Inquiries about Numbers, only as so
many \emph{difficiles nug{\ae}}, so far as they are not
subservient to practice, and promote the benefit of Life.



\paragraph{120.} Unity in Abstract we have before considered in
\emph{Sect.}~13,
from which and what hath been said in the Introduction, it
plainly follows there is not any such Idea.  But Number being
defined a \emph{Collection of Unites}, we may conclude that, if
there be no such thing as Unity or Unite in Abstract, there are
no Ideas of Number in Abstract denoted by the numeral Names and
Figures.  The Theories therefore in Arithmetic, if they are
abstracted from the Names and Figures, as likewise from all Use
and Practice, as well as from the particular things numbered, can
be supposed to have nothing at all for their Object.  Hence we
may see, how intirely the Science of Numbers is subordinate to
Practice, and how jejune and trifling it becomes, when considered
as a matter of mere Speculation.



\paragraph{121.} However since there may be some, who, deluded by the specious
Shew of discovering abstracted Verities, waste their time in
Arithmetical Theoremes and Problemes, which have not any Use: It
will not be amiss, if we more fully consider, and expose the
Vanity of that Pretence; And this will plainly appear, by taking
a view of Arithmetic in its Infancy, and observing what it was
that originally put Men on the Study of that Science, and to what
Scope they directed it.  It is natural to think that at first,
Men, for ease of Memory and help of Computation, made use of
Counters, or in writing of single Strokes, Points or the like,
each whereof was made to signify an Unite, that is, some one
thing of whatever Kind they had occasion to reckon.  Afterwards
they found out the more compendious ways, of making one Character
stand in place of several Strokes, or Points.  And lastly, the
Notation of the \emph{Arabians} or \emph{Indians} came into
use, wherein by the repetition of a few Characters or Figures,
and varying the Signification of each Figure according to the
place it obtains, all Numbers may be most aptly expressed: Which
seems to have been done in Imitation of Language, so that an
exact Analogy is observed betwixt the Notation by Figures and
Names, the nine simple Figures answering the nine first numeral
Names and Places in the former, corresponding to Denominations in
the latter.  And agreeably to those Conditions of the simple and
local Value of Figures, were contrived Methods of finding from
the given Figures or Marks of the Parts, what Figures and how
placed, are proper to denote the whole or \emph{vice versa}.
And having found the sought Figures, the same Rule or Analogy
being observed throughout, it is easy to read them into Words;
and so the Number becomes perfectly known.  For then the Number
of any particular Things is said to be known, when we know the
Name or Figures (with their due arrangement) that according to
the standing Analogy belong to them.  For these Signs being
known, we can by the Operations of Arithmetic, know the Signs of
any Part of the particular Sums signified by them; and thus
computing in Signs, (because of the Connexion established betwixt
them and the distinct multitudes of Things, whereof one is taken
for an Unite,) we may be able rightly to sum up, divide, and
proportion the things themselves that we intend to number.



\paragraph{122.} In \emph{Arithmetic} therefore we regard not the
\emph{Things} but the \emph{Signs}, which nevertheless are
not regarded for their own sake, but because they direct us how
to act with relation to Things, and dispose rightly of them.  Now
agreeably to what we have before observed, of Words in general
(\emph{Sect.}\ 19.\ \emph{Introd.})\ it happens
here likewise, that abstract Ideas are thought to be
signified by Numeral Names or Characters, while they do not
suggest Ideas of particular Things to our Minds.  I shall not at
present enter into a more particular Dissertation on this
Subject; but only observe that it is evident from what hath been
said, those Things which pass for abstract Truths and Theoremes
concerning Numbers, are, in reality, conversant about no Object
distinct from particular numerable Things, except only Names and
Characters; which originally came to be considered, on no other
account but their being \emph{Signs}, or capable to represent
aptly, whatever particular Things Men had need to compute.
Whence it follows, that to study them for their own sake would be
just as wise, and to as good purpose, as if a Man, neglecting the
true Use or original Intention and Subserviency of Language,
should spend his time in impertinent Criticisms upon Words, or
Reasonings and Controversies purely Verbal.



\paragraph{123.} From Numbers we proceed to speak of \emph{Extension}, which
considered as relative, is the Object of Geometry.  The
\emph{Infinite} Divisibility of \emph{Finite} Extension,
though it is not expresly laid down, either as an Axiome or
Theoreme in the Elements of that Science, yet is throughout the
same every where supposed, and thought to have so inseparable and
essential a Connexion with the Principles and Demonstrations in
Geometry, that Mathematicians never admit it into Doubt, or make
the least Question of it.  And as this Notion is the Source from
whence do spring all those amusing Geometrical Paradoxes, which
have such a direct Repugnancy to the plain common Sense of
Mankind, and are admitted with so much Reluctance into a Mind not
yet debauched by Learning: So it is the principal occasion of all
that nice and extreme Subtilty, which renders the Study of
\emph{Mathematics} so difficult and tedious.  Hence if we can
make it appear, that no Finite Extension contains innumerable
Parts, or is infinitely Divisible, it follows that we shall at
once clear the Science of Geometry from a great Number of
Difficulties and Contradictions, which have ever been esteemed a
Reproach to Humane Reason, and withal make the Attainment thereof
a Business of much less Time and Pains, than it hitherto hath
been.



\paragraph{124.} Every particular Finite Extension, which may possibly be the
Object of our Thought, is an \emph{Idea} existing only in the
Mind, and consequently each Part thereof must be perceived.  If
therefore I cannot perceive innumerable Parts in any Finite
Extension that I consider, it is certain they are not contained
in it: But it is evident, that I cannot distinguish innumerable
Parts in any particular Line, Surface, or Solid, which I either
perceive by Sense, or Figure to my self in my Mind: Wherefore I
conclude they are not contained in it.  Nothing can be plainer to
me, than that the Extensions I have in View are no other than my
own Ideas, and it is no less plain, that I cannot resolve any one
of my Ideas into an infinite Number of other Ideas, that is, that
they are not infinitely Divisible.  If by \emph{Finite
Extension} be meant something distinct from a Finite Idea, I
declare I do not know what that is, and so cannot affirm or deny
any thing of it.  But if the terms \emph{Extension},
\emph{Parts}, and the like, are taken in any Sense conceivable,
that is, for Ideas; then to say a Finite Quantity or Extension
consists of Parts infinite in Number, is so manifest a
Contradiction, that every one at first sight acknowledges it to
be so.  And it is impossible it should ever gain the Assent of
any reasonable Creature, who is not brought to it by gentle and
slow Degrees, as a converted Gentile to the belief of
\emph{Transubstantiation}.  Ancient and rooted Prejudices do
often pass into Principles: And those Propositions which once
obtain the force and credit of a \emph{Principle}, are not only
themselves, but likewise whatever is deducible from them, thought
privileged from all Examination.  And there is no Absurdity so
gross, which by this means the Mind of Man may not be prepared to
swallow.



\paragraph{125.} He whose Understanding is prepossest with the Doctrine of
abstract general Ideas, may be persuaded, that (whatever be
thought of the Ideas of Sense,) Extension in \emph{abstract} is
infinitely divisible.  And one who thinks the Objects of Sense
exist without the Mind, will perhaps in virtue thereof be brought
to admit, that a Line but an Inch long may contain innumerable
Parts really existing, though too small to be discerned.  These
Errors are grafted as well in the Minds of
\emph{Geometricians}, as of other Men, and have a like influence
on their Reasonings; and it were no difficult thing, to shew how
the Arguments from Geometry made use of to support the infinite
Divisibility of Extension, are bottomed on them.  At present we
shall only observe in general, whence it is that the
Mathematicians are all so fond and tenacious of this Doctrine.



\paragraph{126.} It hath been observed in another place, that the Theoremes and
Demonstrations in Geometry are conversant about Universal Ideas.
\emph{Sect.}~15.\ \emph{Introd.}
Where it is explained in what Sense this ought to be understood,
to wit, that the particular Lines and Figures included in the
Diagram, are supposed to stand for innumerable others of
different Sizes: or in other words, the Geometer considers them
abstracting from their Magnitude: which doth not imply that he
forms an abstract Idea, but only that he cares not what the
particular Magnitude is, whether great or small, but looks on
that as a thing indifferent to the Demonstration: Hence it
follows, that a Line in the Scheme, but an Inch long, must be
spoken of, as though it contained ten thousand Parts, since it is
regarded not in it self, but as it is universal; and it is
universal only in its Signification, whereby it represents
innumerable Lines greater than it self, in which may be
distinguished ten thousand Parts or more, though there may not be
above an Inch in it.  After this manner the Properties of the
Lines signified are (by a very usual Figure) transferred to the
Sign, and thence through Mistake thought to appertain to it
considered in its own Nature.



\paragraph{127.} Because there is no Number of Parts so great, but it is possible
there may be a Line containing more, the Inch-line is said to
contain Parts more than any assignable Number; which is true, not
of the Inch taken absolutely, but only for the Things signified
by it.  But Men not retaining that Distinction in their Thoughts,
slide into a belief that the small particular Line described on
Paper contains in it self Parts innumerable.  There is no such
thing as the ten-thousandth Part of an \emph{Inch}; but there
is of a \emph{Mile} or \emph{Diameter of the Earth}, which
may be signified by that Inch.  When therefore I delineate a
Triangle on Paper, and take one side not above an Inch, for
Example, in length to be the \emph{Radius}: This I consider as
divided into ten thousand or an hundred thousand Parts, or more.
For though the ten-thousandth Part of that Line considered in it
self, is nothing at all, and consequently may be neglected
without any Error or Inconveniency; yet these described Lines
being only Marks standing for greater Quantities, whereof it may
be the ten-thousandth Part is very considerable, it follows, that
to prevent notable Errors in Practice, the \emph{Radius} must
be taken of ten thousand Parts, or more.



\paragraph{128.} From what hath been said the reason is plain why, to the end any
Theoreme may become universal in its Use, it is necessary we
speak of the Lines described on Paper, as though they contained
Parts which really they do not.  In doing of which, if we examine
the matter throughly, we shall perhaps discover that we cannot
conceive an Inch it self as consisting of, or being divisible
into a thousand Parts, but only some other Line which is far
greater than an Inch, and represented by it.  And that when we
say a Line is \emph{infinitely divisible}, we must mean a Line
which is \emph{infinitely great}.  What we have here observed
seems to be the chief Cause, why to suppose the infinite
Divisibility of finite Extension hath been thought necessary in
Geometry.



\paragraph{129.} The several Absurdities and Contradictions which flowed from this
false Principle might, one would think, have been esteemed so
many Demonstrations against it.  But by I know not what
\emph{Logic}, it is held that Proofs \emph{\`{a} posteriori} are
not to be admitted against Propositions
relating to Infinity.  As though it were not impossible even for
an infinite Mind to reconcile Contradictions.  Or as if any thing
absurd and repugnant could have a necessary Connexion with Truth,
or flow from it.  But whoever considers the Weakness of this
Pretence, will think it was contrived on purpose to humour the
Laziness of the Mind, which had rather acquiesce in an indolent
Scepticism, than be at the Pains to go through with a severe
Examination of those Principles it hath ever embraced for true.



\paragraph{130.} Of late the Speculations about Infinities have run so high, and
grown to such strange Notions, as have occasioned no small
Scruples and Disputes among the Geometers of the present Age.
Some there are of great Note, who not content with holding that
finite Lines may be divided into an infinite Number of Parts, do
yet farther maintain, that each of those Infinitesimals is it
self subdivisible into an Infinity of other Parts, or
Infinitesimals of a second Order, and so on \emph{ad
infinitum}.  These, I say, assert there are Infinitesimals of
Infinitesimals of Infinitesimals, without ever coming to an end.
So that according to them an Inch doth not barely contain an
infinite Number of Parts, but an Infinity of an Infinity of an
Infinity \emph{ad infinitum} of Parts.  Others there be who
hold all Orders of Infinitesimals below the first to be nothing
at all, thinking it with good reason absurd, to imagine there is
any positive Quantity or Part of Extension, which though
multiplied infinitely, can ever equal the smallest given
Extension.  And yet on the other hand it seems no less absurd, to
think the Square, Cube, or other Power of a positive real Root,
should it self be nothing at all; which they who hold
Infinitesimals of the first Order, denying all of the subsequent
Orders, are obliged to maintain.



\paragraph{131.} Have we not therefore reason to conclude, they are \emph{both}
in the wrong, and that there is in effect no such thing as Parts
infinitely small, or an infinite number of Parts contained in any
finite Quantity?  But you will say, that if this Doctrine
obtains, it will follow the very Foundations of Geometry are
destroyed: And those great Men who have raised that Science to so
astonishing an height, have been all the while building a Castle
in the Air.  To this it may be replied, that whatever is useful
in Geometry and promotes the benefit of humane Life, doth still
remain firm and unshaken on our Principles.  That Science
considered as practical, will rather receive Advantage than any
Prejudice from what hath been said.  But to set this in a due
Light, may be the Subject of a distinct Inquiry.  For the rest,
though it should follow that some of the more intricate and
subtile Parts of \emph{Speculative Mathematics} may be pared
off without any prejudice to Truth; yet I do not see what Damage
will be thence derived to Mankind.  On the contrary, it were
highly to be wished, that Men of great Abilities and obstinate
Application would draw off their Thoughts from those Amusements,
and employ them in the Study of such Things as lie nearer the
Concerns of Life, or have a more direct Influence on the Manners.



\paragraph{132.} If it be said that several Theoremes undoubtedly true, are
discovered by Methods in which Infinitesimals are made use of,
which could never have been, if their Existence included a
Contradiction in it, I answer, that upon a thorough Examination
it will not be found, that in any Instance it is necessary to
make use of or conceive infinitesimal Parts of finite Lines, or
even Quantities less than the \emph{Minimum Sensible}: Nay, it
will be evident this is never done, it being impossible.



\paragraph{133.} By what we have premised, it is plain that very numerous and
important Errors have taken their rise from those false
Principles, which were impugned in the foregoing Parts of this
Treatise.  And the Opposites of those erroneous Tenets at the
same time appear to be most fruitful Principles, from whence do
flow innumerable Consequences highly advantageous to true
Philosophy as well as to Religion.  Particularly, \emph{Matter}
or \emph{the absolute Existence of Corporeal Objects}, hath been
shewn to be that wherein the most avowed and pernicious Enemies
of all Knowledge, whether humane or divine, have ever placed
their chief Strength and Confidence.  And surely, if by
distinguishing the real Existence of unthinking Things from their
being perceived, and allowing them a Subsistence of their own out
of the Minds of Spirits, no one thing is explained in Nature; but
on the contrary a great many inexplicable Difficulties arise: If
the Supposition of Matter is barely precarious, as not being
grounded on so much as one single Reason: If its Consequences
cannot endure the Light of Examination and free Inquiry, but
skreen themselves under the dark and general pretence of
\emph{Infinites being incomprehensible}: If withal the Removal
of this \emph{Matter} be not attended with the least evil
Consequence, if it be not even missed in the World, but every
thing as well, nay much easier conceived without it: If lastly,
both \emph{Sceptics} and \emph{Atheists} are for ever silenced
upon supposing only Spirits and Ideas, and this Scheme of Things
is perfectly agreeable both to \emph{Reason} and
\emph{Religion}: Methinks we may expect it should be admitted
and firmly embraced, though it were proposed only as an
\emph{Hypothesis}, and the Existence of Matter had been allowed
possible, which yet I think we have evidently demonstrated that
it is not.



\paragraph{134.} True it is, that in consequence of the foregoing Principles,
several Disputes and Speculations, which are esteemed no mean
Parts of Learning, are rejected as useless.  But how great a
Prejudice soever against our Notions, this may give to those who
have already been deeply engaged, and made large Advances in
Studies of that Nature: Yet by others, we hope it will not be
thought any just ground of Dislike to the Principles and Tenets
herein laid down, that they abridge the labour of Study, and make
Humane Sciences more clear, compendious, and attainable, than
they were before.



\paragraph{135.} Having despatched what we intended to say concerning the
knowledge of \emph{Ideas}, the Method we proposed leads us, in the
next place, to treat of \emph{Spirits}: With regard to which,
perhaps Humane Knowledge is not so deficient as is vulgarly
imagined.  The great Reason that is assigned for our being
thought ignorant of the nature of Spirits, is, our not having an
Idea of it.  But surely it ought not to be looked on as a defect
in a Humane Understanding, that it does not perceive the Idea of
\emph{Spirit}, if it is manifestly impossible there should be any
such \emph{Idea}.  And this, if I mistake not, has been
demonstrated in \emph{Sect.}~27: To which I shall here add that a
Spirit has been shewn to be the only Substance or Support,
wherein the unthinking Beings or Ideas can exist: But that this
\emph{Substance} which supports or perceives Ideas should it
self be an \emph{Idea} or like an \emph{Idea}, is evidently
absurd.



\paragraph{136.} It will perhaps be said, that we want a Sense (as some have
imagined) proper to know Substances withal, which if we had, we
might know our own Soul, as we do a Triangle.  To this I answer,
that in case we had a new Sense bestowed upon us, we could only
receive thereby some new Sensations or Ideas of Sense.  But I
believe no Body will say, that what he means by the terms
\emph{Soul} and \emph{Substance}, is only some particular sort of
Idea or Sensation.  We may therefore infer, that all things duly
considered, it is not more reasonable to think our Faculties
defective, in that they do not furnish us with an Idea of Spirit
or active thinking Substance, than it would be if we should blame
them for not being able to comprehend a \emph{round Square}.



\paragraph{137.} From the opinion that Spirits are to be known after the manner of
an Idea or Sensation, have risen many absurd and heterodox
Tenets, and much Scepticism about the Nature of the Soul.  It is
even probable, that this Opinion may have produced a Doubt in
some, whether they had any Soul at all distinct from their Body,
since upon inquiry they could not find they had an Idea of it.
That an \emph{Idea} which is inactive, and the Existence whereof
consists in being perceived, should be the Image or Likeness of
an Agent subsisting by it self, seems to need no other
Refutation, than barely attending to what is meant by those
Words.  But perhaps you will say, that tho' an \emph{Idea}
cannot resemble a \emph{Spirit} in its Thinking, Acting, or
Subsisting by it self, yet it may in some other respects: And it
is not necessary that an Idea or Image be in all respects like
the Original.



\paragraph{138.} I answer, If it does not in those mentioned, it is impossible it
should represent it in any other thing.  Do but leave out the
Power of Willing, Thinking, and Perceiving Ideas, and there
remains nothing else wherein the Idea can be like a Spirit.  For
by the Word \emph{Spirit} we mean only that which thinks, wills,
and perceives; this, and this alone, constitutes the
Signification of that Term.  If therefore it is impossible that
any degree of those Powers should be represented in an Idea, it
is evident there can be no Idea of a Spirit.



\paragraph{139.}  But it will be objected, that if there is no Idea
signified by the Terms \emph{Soul}, \emph{Spirit}, and
\emph{Substance}, they are wholly insignificant, or have no
meaning in them.  I answer, those Words do mean or signify a real
Thing, which is neither an Idea nor like an Idea, but that which
perceives Ideas, and Wills, and Reasons about them.  What I am my
self, that which I denote by the Term I, is the same with what is
meant by \emph{Soul} or \emph{Spiritual Substance}.  If it be
said that this is only quarrelling at a Word, and that since the
immediate Significations of other Names are by common consent
called \emph{Ideas}, no reason can be assigned, why that which is
signified by the Name \emph{Spirit} or \emph{Soul} may not
partake in the same Appellation, I answer, All the unthinking
Objects of the Mind agree, in that they are intirely passive, and
their Existence consists only in being perceived: Whereas a Soul
or Spirit is an active Being, whose Existence consists not in
being perceived, but in perceiving Ideas and Thinking.  It is
therefore necessary, in order to prevent Equivocation and
confounding Natures perfectly disagreeing and unlike, that we
distinguish between \emph{Spirit} and \emph{Idea}.  See
\emph{Sect.}~27.



\paragraph{140.} In a large Sense indeed, we may be said to have an Idea, or
rather a Notion of \emph{Spirit}, that is, we understand the
meaning of the Word, otherwise we could not affirm or deny any
thing of it.  Moreover, as we conceive the Ideas that are in the
Minds of other Spirits by means of our own, which we suppose to
be Resemblances of them: So we know other Spirits by means of our
own Soul, which in that Sense is the Image or Idea of them, it
having a like respect to other Spirits, that Blueness or Heat by
me perceived has to those Ideas perceived by another.



\paragraph{141.} It must not be supposed, that they who assert the natural
Immortality of the Soul are of opinion, that it is absolutely
incapable of Annihilation even by the infinite Power of the
{\sc Creator} who first gave it Being: But only that it is not liable
to be broken or dissolved by the ordinary Laws of Nature or
Motion.  They indeed, who hold the Soul of Man to be only a thin
vital Flame, or System of animal Spirits, make it perishing and
corruptible as the Body, since there is nothing more easily
dissipated than such a Being, which it is naturally impossible
should survive the Ruin of the Tabernacle, wherein it is
inclosed.  And this Notion hath been greedily embraced and
cherished by the worst part of Mankind, as the most effectual
Antidote against all Impressions of Virtue and Religion.  But it
hath been made evident, that Bodies of what Frame or Texture
soever, are barely passive Ideas in the Mind, which is more
distant and heterogeneous from them, than Light is from Darkness.
We have shewn that the Soul is Indivisible, Incorporeal,
Unextended, and it is consequently Incorruptible.  Nothing can be
plainer, than that the Motions, Changes, Decays, and Dissolutions
which we hourly see befal natural Bodies (and which is what we
mean by the \emph{Course of Nature}) cannot possibly affect an
active, simple, uncompounded Substance: Such a Being therefore is
indissoluble by the force of Nature, that is to say, \emph{the Soul
of Man is naturally immortal}.



\paragraph{142.} After what hath been said, it is I suppose plain, that our Souls
are not to be known in the same manner as senseless inactive
Objects, or by way of \emph{Idea}. \emph{Spirits} and
\emph{Ideas} are Things so wholly different, that when we say,
\emph{they exist}, \emph{they are known}, or the like, these
Words must not be thought to signify any thing common to both
Natures.  There is nothing alike or common in them: And to expect
that by any Multiplication or Enlargement of our Faculties, we
may be enabled to know a Spirit as we do a Triangle, seems as
absurd as if we should hope to \emph{see a Sound}.  This is
inculcated because I imagine it may be of Moment towards clearing
several important Questions, and preventing some very dangerous
Errors concerning the Nature of the Soul.  We may not I think
strictly be said to have an Idea of an active Being, or
of an Action, although we may be said to have a Notion
of them.  I have some Knowledge or Notion of my Mind, and its
Acts about Ideas, inasmuch as I know or understand what is meant
by those Words.  What I know, that I have some Notion of.  I will
not say, that the Terms \emph{Idea} and \emph{Notion} may not
be used convertibly, if the World will have it so.  But yet it
conduceth to Clearness and Propriety, that we distinguish Things
very different by different Names.  It is also to be remarked,
that all Relations including an Act of the Mind, we cannot so
properly be said to have an Idea, but rather a Notion of the
Relations and Habitudes between Things.  But if in the modern way
the word \emph{Idea} is extended to Spirits, and Relations and
Acts; this is after all an affair of verbal Concern.



\paragraph{143.} It will not be amiss to add, that the Doctrine of \emph{Abstract
Ideas} hath had no small share in rendering those Sciences
intricate and obscure, which are particularly conversant about
spiritual Things.  Men have imagined they could frame abstract
Notions of the Powers and Acts of the Mind, and consider them
prescinded, as well from the Mind or Spirit it self, as from
their respective Objects and Effects.  Hence a great number of
dark and ambiguous Terms presumed to stand for abstract Notions,
have been introduced into Metaphysics and Morality, and from
these have grown infinite Distractions and Disputes amongst the
Learned.



\paragraph{144.} But nothing seems more to have contributed towards engaging Men
in Controversies and Mistakes, with regard to the Nature and
Operations of the Mind, than the being used to speak of those
Things, in Terms borrowed from sensible Ideas.  For Example, the
Will is termed the \emph{Motion} of the Soul: This infuses a
Belief, that the Mind of Man is as a Ball in Motion, impelled and
determined by the Objects of Sense, as necessarily as that is by
the Stroke of a Racket.  Hence arise endless Scruples and Errors
of dangerous consequence in Morality.  All which I doubt not may
be cleared, and Truth appear plain, uniform, and consistent,
could but Philosophers be prevailed on to retire into themselves,
and attentively consider their own meaning.



\paragraph{145.} From what hath been said, it is plain that we cannot know the
Existence of other Spirits, otherwise than by their Operations,
or the Ideas by them excited in us.  I perceive several Motions,
Changes, and Combinations of Ideas, that inform me there are
certain particular Agents like my self, which accompany them, and
concur in their Production.  Hence the Knowledge I have of other
Spirits is not immediate, as is the Knowledge of my Ideas; but
depending on the Intervention of Ideas, by me referred to Agents
or Spirits distinct from my self, as Effects or concomitant
Signs.



\paragraph{146.} But though there be some Things which convince us, humane Agents
are concerned in producing them; yet it is evident to every one,
that those Things which are called the Works of Nature, that is,
the far greater part of the Ideas or Sensations perceived by us,
are not produced by, or dependent on the Wills of Men.  There is
therefore some other Spirit that causes them, since it is
repugnant that they should subsist by themselves.  See
\emph{Sect.}~29.
But if we attentively consider the constant Regularity, Order,
and Concatenation of natural Things, the surprising Magnificence,
Beauty, and Perfection of the larger, and the exquisite
Contrivance of the smaller Parts of Creation, together with the
exact Harmony and Correspondence of the whole, but above all, the
never enough admired Laws of Pain and Pleasure, and the Instincts
or natural Inclinations, Appetites, and Passions of Animals; I
say if we consider all these Things, and at the same time attend
to the meaning and import of the Attributes One, Eternal,
infinitely Wise, Good, and Perfect, we shall clearly perceive
that they belong to the aforesaid Spirit, \emph{who works all in
all}, and \emph{by whom all things consist}.



\paragraph{147.} Hence it is evident, that {\sc God} is known as certainly and
immediately as any other Mind or Spirit whatsoever, distinct from
our selves.  We may even assert, that the Existence of {\sc God}
is far more evidently perceived than the Existence of Men;
because the Effects of Nature are infinitely more numerous and
considerable, than those ascribed to humane Agents.  There is not
any one Mark that denotes a Man, or Effect produced by him, which
doth not more strongly evince the Being of that Spirit who is the
\emph{Author of Nature}.  For it is evident that in affecting
other Persons, the Will of Man hath no other Object, than barely
the Motion of the Limbs of his Body; but that such a Motion
should be attended by, or excite any Idea in the Mind of another,
depends wholly on the Will of the {\sc Creator}.  He alone it is
who \emph{upholding all Things by the Word of his Power},
maintains that Intercourse between Spirits, whereby they are able
to perceive the Existence of each other.  And yet this pure and
clear Light which enlightens every one, is it self invisible.



\paragraph{148.} It seems to be a general Pretence of the unthinking Herd, that
they cannot see {\sc God}.  Could we but see him, say they, as we
see a Man, we should believe that he is, and believing obey his
Commands.  But alas we need only open our Eyes to see the
sovereign Lord of all Things with a more full and clear View,
than we do any one of our Fellow-Creatures.  Not that I imagine
we see {\sc God} (as some will have it) by a direct and immediate
View, or see Corporeal Things, not by themselves, but by seeing
that which represents them in the Essence of {\sc God}, which
Doctrine is I must confess to me incomprehensible.  But I shall
explain my Meaning.  A humane Spirit or Person is not perceived
by Sense, as not being an Idea; when therefore we see the Colour,
Size, Figure, and Motions of a Man, we perceive only certain
Sensations or Ideas excited in our own Minds: And these being
exhibited to our View in sundry distinct Collections, serve to
mark out unto us the Existence of finite and created Spirits like
our selves.  Hence it is plain, we do not see a Man, if by {\it
Man} is meant that which lives, moves, perceives, and thinks as
we do: But only such a certain Collection of Ideas, as directs us
to think there is a distinct Principle of Thought and Motion like
to our selves, accompanying and represented by it.  And after the
same manner we see {\sc God}; all the difference is, that whereas
some one finite and narrow Assemblage of Ideas denotes a
particular humane Mind, whithersoever we direct our View, we do
at all Times and in all Places perceive manifest Tokens of the
Divinity: Every thing we see, hear, feel, or any wise perceive by
Sense, being a Sign or Effect of the Power of {\sc God}; as is
our Perception of those very Motions, which are produced by Men.



\paragraph{149.} It is therefore plain, that nothing can be more evident to any
one that is capable of the least Reflexion, than the Existence of
{\sc God}, or a Spirit who is intimately present to our Minds,
producing in them all that variety of Ideas or Sensations, which
continually affect us, on whom we have an absolute and intire
Dependence, in short, \emph{in whom we live, and move, and have
our Being}.  That the Discovery of this great Truth which lies so
near and obvious to the Mind, should be attained to by the Reason
of so very few, is a sad instance of the Stupidity and
Inattention of Men, who, though they are surrounded with such
clear Manifestations of the Deity, are yet so little affected by
them, that they seem as it were blinded with excess of Light.



\paragraph{150.} But you will say, Hath Nature no share in the Production of
natural Things, and must they be all ascribed to the immediate
and sole Operation of {\sc God}? I answer, If by \emph{Nature}
is meant only the visible \emph{Series} of Effects, or
Sensations imprinted on our Minds according to certain fixed and
general Laws: Then it is plain, that Nature taken in this Sense
cannot produce any thing at all.  But if by \emph{Nature} is
meant some Being distinct from {\sc God}, as well as from the
Laws of Nature, and Things perceived by Sense, I must confess
that Word is to me an empty Sound, without any intelligible
Meaning annexed to it.  Nature in this Acceptation is a vain
\emph{Chimera} introduced by those Heathens, who had not just
Notions of the Omnipresence and infinite Perfection of {\sc God}.
But it is more unaccountable, that it should be received among
\emph{Christians} professing Belief in the Holy Scriptures,
which constantly ascribe those Effects to the immediate Hand of
{\sc God}, that Heathen Philosophers are wont to impute to
\emph{Nature}.  \emph{The LORD, he causeth the Vapours to ascend;
he maketh Lightnings with Rain; he bringeth forth the Wind out of
his Treasures}, Jerem.\ Chap.~10.\ ver.~13.  \emph{He turneth the
shadow of Death into the Morning, and maketh the Day dark with
Night}, Amos Chap.~5.\ ver.~8.  \emph{He visiteth the Earth, and
maketh it soft with Showers: He blesseth the springing thereof,
and crowneth the Year with his Goodness; so that the Pastures are
clothed with Flocks, and the Valleys are covered over with Corn}.
See \emph{Psalm}~65.  But notwithstanding that this is the
constant Language of Scripture; yet we have I know not what
Aversion from believing, that {\sc God} concerns himself so
nearly in our Affairs.  Fain would we suppose him at a great
distance off, and substitute some blind unthinking Deputy in his
stead, though (if we may believe Saint \emph{Paul}) \emph{he be
not far from every one of us}.



\paragraph{151.} It will I doubt not be objected, that the slow and gradual
Methods observed in the Production of natural Things, do not seem
to have for their Cause the immediate Hand of an \emph{almighty
Agent}.  Besides, Monsters, untimely Births, Fruits blasted in
the Blossom, Rains falling in desert Places, Miseries incident to
humane Life, are so many Arguments that the whole Frame of Nature
is not immediately actuated and superintended by a Spirit of
infinite Wisdom and Goodness.  But the Answer to this Objection
is in a good measure plain from \emph{Sect.}~62, it being visible,
that the aforesaid Methods of Nature are absolutely necessary, in
order to working by the most simple and general Rules, and after
a steady and consistent Manner; which argues both the
\emph{Wisdom} and \emph{Goodness} of {\sc God}.  Such is the
artificial Contrivance of this mighty Machine of Nature, that
whilst its Motions and various Ph{\ae}nomena strike on our
Senses, the Hand which actuates the whole is it self
unperceivable to Men of Flesh and Blood.  \emph{Verily} (saith
the Prophet) \emph{thou art a GOD that hidest thy self}, Isaiah
Chap.~45.\ ver.~15.  But though {\sc God} conceal himself from
the Eyes of the \emph{Sensual} and \emph{Lazy}, who will not be
at the least Expense of Thought; yet to an unbiassed and
attentive Mind, nothing can be more plainly legible, than the
intimate Presence of an \emph{All-wise Spirit}, who fashions,
regulates, and sustains the whole Systeme of Being.  It is clear
from what we have elsewhere observed, that the operating
according to general and stated Laws, is so necessary for our
Guidance in the Affairs of Life, and letting us into the Secret
of Nature, that without it, all Reach and Compass of Thought, all
humane Sagacity and Design could serve to no manner of purpose:
It were even impossible there should be any such Faculties or
Powers in the Mind.  See \emph{Sect.}~31.  Which one Consideration
abundantly out-balances whatever particular Inconveniences may
thence arise.



\paragraph{152.} We should further consider, that the very Blemishes and Defects
of Nature are not without their Use, in that they make an
agreeable sort of Variety, and augment the Beauty of the rest of
the Creation, as Shades in a Picture serve to set off the
brighter and more enlightened Parts.  We would likewise do well
to examine, whether our taxing the Waste of Seeds and Embryos,
and accidental Destruction of Plants and Animals, before they
come to full Maturity, as an Imprudence in the Author of Nature,
be not the effect of Prejudice contracted by our Familiarity with
impotent and saving Mortals.  In \emph{Man} indeed a thrifty
Management of those Things, which he cannot procure without much
Pains and Industry, may be esteemed \emph{Wisdom}.  But we must
not imagine, that the inexplicably fine Machine of an Animal or
Vegetable, costs the great {\sc Creator} any more Pains or
Trouble in its Production than a Pebble doth: nothing being more
evident, than that an omnipotent Spirit can indifferently produce
every thing by a mere \emph{Fiat} or Act of his Will.  Hence it
is plain, that the splendid Profusion of natural Things should
not be interpreted, Weakness or Prodigality in the Agent who
produces them, but rather be looked on as an Argument of the
Riches of his Power.



\paragraph{153.} As for the mixture of Pain or Uneasiness which is in the World,
pursuant to the general Laws of Nature, and the Actions of finite
imperfect Spirits: This, in the State we are in at present, is
indispensably necessary to our well-being.  But our Prospects are
too narrow: We take, for Instance, the Idea of some one
particular Pain into our Thoughts, and account it \emph{Evil};
whereas if we enlarge our View, so as to comprehend the various
Ends, Connexions, and Dependencies of Things, on what Occasions
and in what Proportions we are affected with Pain and Pleasure,
the Nature of humane Freedom, and the Design with which we are
put into the World; we shall be forced to acknowledge that those
particular Things, which considered in themselves appear to be
\emph{Evil}, have the Nature of \emph{Good}, when considered
as linked with the whole Systeme of Beings.



\paragraph{154.} From what hath been said it will be manifest to any considering
Person, that it is merely for want of Attention and
Comprehensiveness of Mind, that there are any Favourers of
\emph{Atheism} or the \emph{Manichean Heresy} to be found.
Little and unreflecting Souls may indeed burlesque the Works of
Providence, the Beauty and Order whereof they have not Capacity,
or will not be at the Pains to comprehend.  But those who are
Masters of any Justness and Extent of Thought, and are withal
used to reflect, can never sufficiently admire the divine Traces
of Wisdom and Goodness that shine throughout the Oeconomy of
Nature.  But what Truth is there which shineth so strongly on the
Mind, that by an Aversion of Thought, a wilful shutting of the
Eyes, we may not escape seeing it?  Is it therefore to be
wondered at, if the generality of Men, who are ever intent on
Business or Pleasure, and little used to fix or open the Eye of
their Mind, should not have all that Conviction and Evidence of
the Being of {\sc God}, which might be expected in reasonable
Creatures?



\paragraph{155.} We should rather wonder, that Men can be found so stupid as to
neglect, than that neglecting they should be unconvinced of such
an evident and momentous Truth.  And yet it is to be feared that
too many of Parts and Leisure, who live in Christian Countries,
are merely through a supine and dreadful Negligence sunk into a
sort of \emph{Atheism}.  Since it is downright impossible, that a
Soul pierced and enlightened with a thorough Sense of the
Omnipresence, Holiness, and Justice of that \emph{Almighty
Spirit}, should persist in a remorseless Violation of his Laws.
We ought therefore earnestly to meditate and dwell on those
important Points; that so we may attain Conviction without all
Scruple, \emph{that the Eyes of the LORD are in every place
beholding the Evil and the Good; that he is with us and keepeth
us in all places whither we go, and giveth us Bread to eat, and
Raiment to put on}; that he is present and conscious to our
innermost Thoughts; and that we have a most absolute and
immediate dependence on him.  A clear View of which great Truths
cannot choose but fill our Hearts with an awful Circumspection
and holy Fear, which is the strongest Incentive to \emph{Virtue},
and the best Guard against \emph{Vice}.



\paragraph{156.} For after all, what deserves the first place in our Studies, is
the Consideration of \emph{GOD}, and our \emph{Duty}; which to
promote, as it was the main drift and design of my Labours, so
shall I esteem them altogether useless and ineffectual, if by
what I have said I cannot inspire my Readers with a pious Sense
of the Presence of {\sc God}: And having shewn the Falseness or
Vanity of those barren Speculations, which make the chief
Employment of learned Men, the better dispose them to reverence
and embrace the salutary Truths of the {\sc Gospel}, which to
know and to practice is the highest Perfection of humane Nature.

\end{sectionbody}

\end{document}
